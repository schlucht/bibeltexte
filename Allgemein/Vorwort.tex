\section{Allgemein}
Janina und ich haben uns entschieden einen Bibelkurs bei der Freien Evangelischen Kirche Mitternachtsruf durch zu führen. Durch diesen Kurs erhoffen wir, dass wir eine bessere Beziehung zu Jesus aufbauen können.

\subsection{Mitternachtsruf}
Der Mitternachtsruf \footnote{www.mnr.ch} ist ein freies evangelisches Missionswerk mit dem Ziel, die Menschen auf Jesus Christus, Seine frohe Botschaft und Seine Rückkehr auf diese Erde hinzuweisen. Ihr Name leitet sich ab \begin{bibeltext}{Sch2}{Matt}{25:6}
    			Um Mitternacht aber entstand ein Geschrei: \glqq Siehe der Bräutigam kommt! Geht aus, ihm entgegen!\grqq{}
    		\end{bibeltext}
Diese Gemeinde bietet eine Bibelschule\footnote{Gemeinde Bibel Schule Mitternachtsruf (GBSM)} an, die gratis ist und 2 Jahre dauert. 

\subsection{Themen}
\begin{itemize}
    \item \textbf{Aionologie: }die Lehre von den Zeitaltern. was sagt die Bibel über die verschiedenen Zeitalter in Gottes Heilsplan?
    \item \textbf{Anthropologie: }die Lehre von dem Menschen.
    Was sagt die Bibel über Ursprung, Ziel und Wesen des Menschen?
    \item \textbf{Bibiologie: }die Lehre des Buches. Wie ist die Bibel entstanden und was sagt sie über sich selbst aus?
    \item \textbf{Biblische Geografie: }die Kunde des Landes der Bibel. Wo befinden sich und welche Rolle spielen die historischen Orte der Bibel?
    \item \textbf{Ekkesiologie: }die Lehre von der Gemeinde. Was sagt die Bibel über die Gemeinde des lebendigen Gottes?
    \item \textbf{Eschatologie: }die Lehre von den letzten Dingen. Was sagt die Bibel über die Zukunft des Menschen und der Welt?
    \item \textbf{Hamartiologie: }Hamartiologie: die Lehre von der Sünde. Was sagt die Bibel über das Wesen und die
    Auswirkungen der Sünde?
    \item \textbf{Israelogie: }die Lehre von Israel. Was sagt die Bibel über das Volk Israel?
    \item \textbf{Missionologie: }die Lehre von der Mission. Was sagt die Bibel über den Missionsauftrag der Christen?
    \item \textbf{Theologie: }die Lehre von Gott. Was sagt die Bibel über Gott selbst und wie hat sich diese Lehre in der Gemeinde entwickelt?
\end{itemize}
\subsection{Autoren}
\begin{itemize}
    \item Norbert Lieth
    \item Thomas Lieth
    \item Fredy Peter
    \item Nathanael Winkler
    \item Samuel Rindlisbacher
    \item René Malgo
\end{itemize}
\subsection{Ort}
\parbox{3.5in}{\textbf{Maranatha Haus}}\\
\parbox{3.5in}{Zionsweg 1} \\
\parbox{3.5in}{8600 Dübendorf} \\
\parbox{3.5in}{Webseite: www.gbsm.ch}