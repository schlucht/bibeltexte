\subsection{Kapitel 6}
Am Anfang des Kapitels wird erklärt, warum Gott die Menschen vernichten wollte. Es steht nirgends, dass er die Erde vernichten will, sondern nur die Lebewesen auf dieser Erde. Gott fand, das alle Menschen schlecht sind. In Vers 6 steht folgendes.
\begin{bibeltext}{ELB}{Gen}{6:6}
	Und es reute den \herr , dass er den Menschen gemacht hatte auf der Erde, und es schmerzte ihn in sein Herz hinein.
\end{bibeltext}
In Vers 7 steht, dass er nicht nur die Menschen vernichten will sondern auch alle Tier auf dieser Erde. Und dann in Vers 8.
\begin{bibeltext}{ELB}{Gen}{6:7}
	Noah aber fand Gnade in den Augen des Herrn.
\end{bibeltext}
Auf der ganzen Erde nur ein Mann. Dieser eine Mann wird als gerechter und vollkommener Mann unter seinen Zeitgenossen dargestellt. Durch diesen gerechten Menschen wurden dann auch die Tier nach ihrer Art gerettet. Für die Rettung der Menschen und der Tiere gab ihm Gott den Auftrag ein Schiff zu bauen. Mit dem Schiff und dem Regen fängt eine Geschichte an, die mit der Wiederkunft Jesus im neuen Testament vergleichbar ist. Nur mit einer Sintflut und einem grossen Schiff, kann so eine parallele zu unserer heutigen Zeit unsere Erlösung mit Jesus Christus bildlich dargestellt werden.
\begin{figure}[ht]
	\begin{tabular}{lcl}
		Schiff & = & Jesus            \\
		Wasser & = & ungläuber Mensch \\
		Noah   & = & gläubige Mensch  \\
	\end{tabular}
\end{figure}
Noah hat Gott gehorcht und das Schiff gebaut. Die restliche Menschheit wollte sich nicht von einem Gott etwas sagen lassen. Da sind wir schlauer. Uns geht es ja gut, für was brauchen wir noch einen Gott? Sollen wir und die Mühe machen und ein Schiff zusammen zimmern? Läuft doch alles so prima. Eines Tages fing es an zu regnen. Naja ein bisschen Regen wieso sich Sorgen machen. Der Regen wurde aber immer schlimmer.

Das passiert doch auch heute. Mehr Überschwemmungn, mehr Krankheiten, mehr Unruhen und Unzufriedenheit in der Menschheit. Perspektivlosigkeit, werden mit Versuchen den Weltraum zu ergründen überbrückt. Von Menschen gemachter Klimawandel? Vielleicht ist es auch ein von Gott gemachter Klimawandel und es geht wie in der Offenbarung beschrieben langsam dem Ende entgegen? Macht man sich die Mühe mit Jesus Christus unterwegs zu sein, wird man ausgelacht. \textbf{Wir} haben alles im Griff. \textbf{Wir} brauchen niemand, ein bisschen CO\textsuperscript{2} weniger in die Luft jagen und schon wird es uns wieder besser gehen. Kann sein, dass es so wird, muss aber nicht. Es kann aber auch wie zu Noahs Zeiten werden, dass wir besser auf Jesus vertrauen sollten und so in die Arche gelangen um gerettet zu werden.
\subsection{Kapitel 7}
In diesem Kapitel geht es in die Arche. Jetzt fordert Gott Noah auf das Schiff zu füllen und selber in die Arche zu steigen. Gott schickt ein Pärchen von jeder Tierart.

Mit Noah und den Tieren, steigen auch seine Frau, seine 3 Söhne Sem, Ham und Japhet mit ihren Frauen in die Arche ein. Interessant ist, dass Gott im Vers 2 schon hier von reinen und unreinen Tieren spricht. Von den reinen soll er je 7 von jeder Tierart und von den Männchen und Weibchen mitnehmen. Von den unreinen je ein Pärchen. Welche Tiere rein und unrein waren wird nicht verraten. Von den reinen Tieren wurde dann in \biblezit{Gen}{8:20} ein Opfer für den \herr n dargebracht.

Noah war 600 Jahre und zwei Monate alt als das Unwetter begann. Es wird nicht nur Regen beschrieben, sondern das Wasser kam von unten und von Oben. Es taten sich alle Brunnen in der grossen Tiefe auf. Der Regen kam dann noch zusätzlich. Es waren riesige Massen an Wasser in kurzer Zeit, welche die Erde überschwemmte. Alle Menschen und Tiere auf dem Land kamen ums leben. Allein Noah und die Passagiere auf der Arche überlebten dieses Inferno.

So wird es auch in den letzten Tagen sein. Viele werde in den letzten Tagen draussen stehen und verloren gehen. Sie werden überrascht sein. Sie werden wütend sein. Trotzdem werden sie nicht an Gott glauben. Sie werden das Unwetter allem anderen zuschieben nur nicht Gott und an seine Rettung glauben. Aus Frust werden sie die verfolgen die an Jesus Christus als den Erlöser glauben. Sie werden sehen, dass diese etwas an sich haben, was sie beruhigt. Wie viele sind der Arche nach geschwommen? Wie viele wollten nachträglich noch auf das Schiff klettern? Viele wurden überrascht. Viele wussten nichts von dem Schiff oder haben das wohl als Geschichte oder Märchen abgetan. Als der Regen anfing haben sie das wohl einem Klimawandel zugeschrieben. Ist unser Klimawandel auch Zeichen Gottes? Bestimmt, es fällt uns ja kein Haar vom Kopf ohne dass er davon weiss.

Das Wasser blieb 150 Tage hoch über der Erde. Es heisst, das Wasser stieg 15 Ellen über den höchsten Berg der Erde. Da war kein entkommen für die Landlebewesen. Alles auf der Erde wurde getötet. Das interessante ist, dass alle Kulturen auf dieser Erde ein Sintflut kennen.

Das Sterben im Wasser war sicher nicht angenehm, aber da alle Menschen auf der Erde gestorben sind gab es auch kein Leid. Das Leid entsteht bei den Zurückgebliebenen, bei diesen ist der Schmerz des alleine seins und verlassen sein gross. Wenn alle Tod sind passiert das nicht. Es ist die Hoffnung der Christen sich eines Tages wieder zu sehen. Dies mildert der Schmerz. Und dass die Person die gestorben ist, es jetzt besser hat als im Leben. Wenn man daran glaubt, dass einen die Würmer fressen kann schon verzweifeln.
\subsection{Kapitel 8}
Zu Beginn des Regens war Noah 600 Jahre alt. Noah war 371 Tage auf diesem Schiff, bis der \herr ihm befohlen das Schiff wieder zuj v
