
\subsection{Kapitel 2}
Im zweiten Kapitel wird am Anfang der 7. Tag der Woche beschrieben.
\begin{bibeltext}{Sch2}{Gen}{2:3}
	Und Gott segnete den siebten Tag und heiligte ihn, denn an ihm ruhte er von seinem ganzen Werk, das Gott schuf, als er es machte.
\end{bibeltext}
Hier steht ja nirgends, dass der siebte Tag ein Sonntag oder ein Sabbat war. Es heisst einfach, dass Gott nach 6 Tagen arbeiten, diese hingelegt hat. Am siebten Tag hat er geruht. Er ging nicht auf den neuen Meer Surfen, sondern er hat geruht. Welcher Tag das in der Woche ist finde ich unwichtig. Aber es sollte ein Tag in der Woche geruht werden.

An diesem Tag können wir einen Rückblick der Woche machen und Gott für seine Fürsorge danken. In unserem Arbeitssystem ist sicher gestellt, dass wir einen Tag frei haben. Öfters wird dieser Tag aber für einen andere Arbeit benutzt. Es gibt sicher Menschen denen das Geld von sechs Tage Arbeit, zum leben nicht reicht. Es geht aber nicht nur um die Existenz. Bei uns im Wallis war Sonntagsarbeit immer verpönt. Die Nachbarn haben darauf geachtet, dass man nicht Arbeit und zum Gottesdienst geht. Die jungen Bauern von heute kennen das nicht mehr. Die Woche auf der regulären Arbeit, am Wochenende dann auf der Wiese. Aber viele Hobbys arten zu Arbeit aus. Vor allem Sport Wettkampf. Das Training die Leistung alles braucht Substanz und man kommt nicht zur der Ruhe die man braucht um sich zu erholen.

In den restlichen Versen wird dann die Erschaffung des Menschen detaillierter aufgezeigt. Von 4--6 wird nochmals kurz die Schöpfung zusammengefasst und dann die Erschaffung den Menschen aus \textquote{Staub von der Erde}.

Innerhalb seiner Schöpfung pflanzte Gott einen Garten und setzte den Menschen da hinein. In diesem Garten standen auch die zwei verhängnisvollen Bäume \textquote{Baum des Lebens} und \textquote{Baum der Erkenntnis des Guten und Bösen}. Dieser Garten wird der Garten Eden genannt. In diesem Garten lebte der Mensch mit Gott zusammen. Gott hat die Angewohnheit am Abend in Eden spazieren zu gehen. Gott gab aber dem Menschen die Anweisung, dass er die Früchte von dem \textquote{Baum der Erkenntnis des Guten und Bösen}, nicht essen darf.
\begin{bibeltext}{Sch2}{Gen}{2:17}
	aber von dem Baum der Erkenntnis des Guten und des Bösen sollst du nicht essen; denn an dem Tag, da du davon isst, musst du gewisslich sterben!\footnote{Die ersten Menschen kannten den Tod noch nicht; er kam erst als Folge der Sünde über den Menschen. \bibleverse{Rom} {5:12}; \bibleverse{Rom} {6:23}; \bibleverse{Eph} {2:1-3}}
\end{bibeltext}

Ab Vers 18 kommen wir zur Erschaffung der Frau. Nachdem der erste Mensch allen Tieren einen Namen gegeben hat und herausgefunden hatte, dass unter den Tieren kein wirklich gegenüber zu finden war, hat Gott aus der Rippe des ersten Menschen die Frau erschaffen.
\begin{bibeltext}{Sch2}{Gen}{2:24}
	Darum wird ein Mann seinen Vater und seine Mutter verlassen und seiner Frau anhängen.
\end{bibeltext}
Ein interessanter Satz. Das zeigt mir, dass diese ersten beiden Menschen auch im Garten Eden Kinder bekommen hätten und diese dann später geheiratet hätten. Oder der Autor hat diesen Satz hinzugefügt, um die heutige Hochzeit zwischen Mann und Frau zu erklären. In jeder Kultur werden die Paare verheiratet. Irgend wie hat das Heiraten was. Wieso wollen gleichgeschlechtliche Paare eigentlich immer heiraten? Wieso reicht es ihnen nicht einfach zusammen zu leben? Gesetzlich könnte man das ja einfach regeln. Es muss etwas im Menschen sein das diesem das Bedürfnis gibt, dass eine höhere Instanz die Zustimmung zu dem zusammenleben gibt.
