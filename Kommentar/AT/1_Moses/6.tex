\subsection{Kapitel 6}
Am Anfang des Kapitels wird erklärt, warum Gott die Menschen vernichten wollte. Es steht nirgends, dass er die Erde vernichten will, sondern nur die Lebewesen auf dieser Erde. Gott fand, das alle Menschen schlecht sind. In Vers 6 steht folgendes.
\begin{bibeltext}{ELB}{Gen}{6:6}
	Und es reute den \herr , dass er den Menschen gemacht hatte auf der Erde, und es schmerzte ihn in sein Herz hinein.
\end{bibeltext}
In Vers 7 steht, dass er nicht nur die Menschen vernichten will sondern auch alle Tier auf dieser Erde. Und dann in Vers 8.
\begin{bibeltext}{ELB}{Gen}{6:7}
	Noah aber fand Gnade in den Augen des Herrn.
\end{bibeltext}
Auf der ganzen Erde nur ein Mann. Dieser eine Mann wird als gerechter und vollkommener Mann unter seinen Zeitgenossen dargestellt. Durch diesen gerechten Menschen wurden dann auch die Tier nach ihrer Art gerettet. Für die Rettung der Menschen und der Tiere gab ihm Gott den Auftrag ein Schiff zu bauen. Mit dem Schiff und dem Regen fängt eine Geschichte an, die mit der Wiederkunft Jesus im neuen Testament vergleichbar ist. Nur mit einer Sintflut und einem grossen Schiff, kann so eine parallele zu unserer heutigen Zeit unsere Erlösung mit Jesus Christus bildlich dargestellt werden.
\begin{figure}[ht]
	\begin{tabular}{lcl}
		Schiff & = & Jesus\\
		Wasser & = & ungläuber Mensch\\
		Noah & = & gläubige Mensch\\
	\end{tabular}
\end{figure}
Noah hat Gott gehorcht und das Schiff gebaut. Die restliche Menschheit wollte sich nicht von einem Gott etwas sagen lassen. Da sind wir schlauer. Uns geht es ja gut, für was brauchen wir noch einen Gott? Sollen wir und die Mühe machen und ein Schiff zusammen zimmern? Läuft doch alles so prima. Eines Tages fing es an zu regnen. Naja ein bisschen Regen wieso sich Sorgen machen. Der Regen wurde aber immer schlimmer.

Das passiert doch auch heute. Mehr Überschwemmungn, mehr Krankheiten, mehr Unruhen und Unzufriedenheit in der Menschheit. Perpektivlosigkeiten, werden mit Versuchen den Weltraum zu ergründen überbrückt. Von Menschen gemachter Klimawandel? Vielleicht ist es auch ein von Gott gemachter Klimawandel und es geht wie in der Offenbarung beschrieben langsam dem Ende entgegen? Macht man sich die Mühe mit Jesus Christus unterwegs zu sein, wird man ausgelacht. \textbf{Wir} haben alles im Griff. \textbf{Wir} brauchen niemand, ein bisschen CO\textsuperscript{2} weniger in die Luft jagen und schon wird es uns wieder besser gehen. Kann sein, dass es so wird, muss aber nicht. Es kann aber auch wie zu Noahs Zeiten werden, dass wir besser auf Jesus vertrauen sollten und so in die Arche gelangen um gerettet zu werden.