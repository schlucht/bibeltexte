\subsection{Kapitel 8}
Zu Beginn des Regens war Noah 600 Jahre alt. Noah war 371 Tage auf diesem Schiff, bis der \herr ihm befohlen das Schiff wieder zu verlassen. Es war eine lange Zeit auf dem Schiff und sicher nicht angenehm. Die psychische Belastung, der Gestank und es war sicher auch laut mit diesen ganzen Tieren. So ist es auch verständlich, dass Noah ungedultig wurde und schon mal einen Raben ausschickte um eine Prüfung der Verhältnisse vor zu nehmen. Als dann der Rabe nicht mehr zurück kam, hat er dann eine Taube im Wochenrhythmus ausgeschickt. Als die dann mit einem Ölblatt zurück kam, konnte er sich vorstellen, dass es wohl nicht mehr so lange geht. Es ging aber noch zwei Monate bis Gott ihn aus der Arche liess.

Es ist doch bei uns auch so. Wir werden sehr Ungeduldig wenn wir in Bedrängnis oder in Not sind. Ist ja auch verständlich. Wer hat schon gerne wenn es ihm schlecht geht? Die folgen für Noah wären fatal gewesen, wäre er zu früh aus dem Schiff gestiegen. Er hätte noch nicht überleben können. Wenn wir Ungeduldig werden und selber vorwärts machen wollen kann das schief gehen. Gott kennt den genauen Zeitpunkt wann er uns wieder auf die Welt los lassen kann ohne das wir schaden davon nehmen. So auch bei Noah. Als Gott ihm sagte, du kannst jetzt raus, war die Welt wieder so hergestellt, dass Noah, seine Familie und die Tiere wieder ohne Gefahr leben konnten. Diese Geduld müssen wir lernen. In unserer schnelllebigen Zeit ist es noch viel schwieriger die Geduld zu waren. Aber um so wichtiger. Es heisst ja nicht, dass wir nicht fragen können, oder informationen sammeln. Der Herr war nicht beleidigt als Moses sein Vögel ausgesandt hat. Er kennt unsere Herzen und weiss genau aus welchem Grund wir irgendwas machen. Aber, wir sollten auf die Freigabe des Herrn warten, bevor wir aus dem Schiff aussteigen. Der Herr gab Noah den Aufrag Fruchtbar zu sein und die Erde zu bevölkern. Auch die Apostel und jetzt wir bekamen diesen Auftrag. Bei Himmelfahrt Jesus sagte er zu den Aposteln. 
\begin{bibeltext}{Elb}{Apg}{1:8}
	Aber ihr werdet Kraft empfangen, wenn der Heilige Geist auf euch gekommen ist; und ihr werdet meine Zeugen sein, sowohl in Jerusalem als auch in ganz Judäa und Samaria und bis an das Ende der Erde.
\end{bibeltext} 
Die Apostel mussten auch eine Durststrecke durchleben nach dem Tod von Jesus. Sie brauchten auch Gedult und Jesus hat sie 40 Tage gelehrt. Danach war der Weg zwar immer noch steinig, aber durch die Hilfe des Heiligen Geistes kamen sie durch alle Berdrängnis.

Was hat Noah als erstes gemacht? Er hat Gott einen Altar gebaut und im Dankopfer dargebracht. Er hat sich nicht auf den Boden gesetzt und mit seinem Schicksal gehadert. Er wusste mit Gott wird das schon gehen. Er hat Gott gedankt, für sein Treue und Gnade ihm und seiner Familie gegenüber. Wie oft danken wir auch wenn es uns nicht so gut geht? Trozdem wir keinen Ausgang mehr sehen?