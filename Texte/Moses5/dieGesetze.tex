\section{Das Bücher 5. Mose}
\subsection{Intressante Gesetze}

Das Buch 5. Mose oder auch Deuteronomium, wiederholt die von Gott gegeben Gesetze in Moses 2. Nur diesemal in einer Form die ich, wie ich finde verständlicher finde.\\
Mose liest oder erzählt diese Gebote dem Volk nochmals, bevor diese das von Gott verheissene Land betreten.\\
Das spezielle an diesem Buch ist, dass hier immer wieder zum Vorschein kommt, dass Gott will, dass wir fröhlich sind und das Leben geniessen. Diese Fröhlichkeit und das geniessen, soll aber nicht auf dem Buckel unserer Mitmenschen ausgeführt werden. \\
\begin{bibeltext}{Sch2}{5Mos}{15:4}
	\textsuperscript{4} Es soll zwar unter euch gar kein Armer sein; denn der \textsc{Herr} wird dich reichlich segnen...
\end{bibeltext}
Das heisst es soll mehr oder weniger allen gut gehen, aber er weiss , dass es anders kommen wird. In diesen Gesetzen sind viele Gebote die ein friedliches zusammenleben einfach machen. Daneben sind aber auch Gesetze und Strafen die wir in der heutigen Zeit als brutal und barbarisch ansehen. So wird auch immer wieder gesagt wieso diese Strafe so hart ist. \flqq ... und \textbf{du} sollt das Böse aus deiner Mitte ausrotten.\frqq{} oder \flqq \textbf{So} sollst du das Böse aus deiner Mitte ausrotten.\frqq{}
Es geht also darum die Bösen die vorsätzlich handeln aus dem Verkehr zu ziehen.\\
Eigentlich fängt es schon im Kapitel 9 an. Dort holt der \textsc{Herr} die Leute wieder von ihren hohen Ross runter, was auch uns in der heutigen Zeit gut tun würde.
\begin{bibeltext}{Sch2}{5Mos}{9:5-6}
    \textsuperscript{5} Denn nicht um deiner Gerechtigkeit und um deines aufrichtigen Herzens willen kommst du hinein, um ihr Land in den Besitz zu nehmen, sondern wegen ihrer Gottlosigkeit vertreibt der \textsc{Herr}, dein Gott, diese Heidenvölker aus ihrem Besitz, und damit er das Wort aufrechterhalte, das der \textsc{Herr} deinen Väter Abraham, Isaak und Jakob geschworen hat.\\
    \textsuperscript{6} So sollst du nun erkennen, dass der \textsc{Herr}, dein Gott, dir dieses gute Land nicht um deiner Gerechtigkeit willen gibt,damit du es in Besitz nimmst; denn du bist ein halsstarriges Volk!
\end{bibeltext}
Wie sieht das mit uns in Europa aus? Sind wir jetzt das halsstarrige Volk? Oder ist einfach nur Zufall das es uns so gut geht? Ich sehe das so: Europa ist das Jordanland und die gottestreuen Gemeinden ist Israel.
Da fragt man sich, wieso geht uns eigentlich so gut? Sollten wir eigentlich nicht auch ausgerottet werden? Oder haben unser Vorfahren hier in Europa doch etwas gut beim \textsc{Herrn}?\\
Die ganzen Kriege schon im finstern Mittelalter. War die Pest schon mal ein Fingerzeig Gottes. Wir haben es uns schon zum Sport gemacht, sobald es uns gut geht, probieren wir das Leben in die eigenen Hände zu nehmen und schicken den \textsc{Herr} in die Wüste.\\
Steht uns das Wasser aber bis an den Kragen, dann solle Gute aber sofort wieder kommen und helfen. Das wusste Gott schon damals darum sagt:
\begin{bibeltext}{Sch2}{5Mos}{6:10-13}
    \textsuperscript{10} Wenn dich nun der \textsc{Herr}, dein Gott, in das Land bringen wird, von dem er deinen Vätern Abraham, Isaak und Jakob geschworen hat, es dir zu geben, grosse und gut Städte, die du nicht gebaut hast, \textsuperscript{11} und Häuser, voll von allem Guten, die du nicht gefüllt hast, und ausgehauene Zisternen, die du nicht ausgehauen hast, Weinberge und Ölbäume, die du nicht gepflanzt hast; und wenn du isst und satt geworden bis, \textsuperscript{12} so hüte dich davor, den \textsc{Herrn} zu vergessen, der dich aus dem Land Ägypten, aus dem Haus der Knechtschaft, herausgeführt hat; \textsuperscript{13} sondern du sollst den \textsc{Herrn}, deinen Gott, fürchten und ihm dienen und bei seinem Namen schwören.
\end{bibeltext}
Dienen wir IHM heute noch? Schwören wir bei seinem Namen? Wird dienen dem Geld und schwören auf unsere Ethik.\\
Auf diese Ethik sind wir besonders stolz. Wir sind ja nicht so brutal wie der Gott in der Bibel der sagt:
\begin{bibeltext}{Sch2}{5Mos}{22:25}
    \textsuperscript{25} Wenn aber der Mann das verlobte Mädchen auf dem Feld antrifft und sie ergreift und bei ihr liegt, so soll der Mann, der bei ihr liegt alleine sterben.
\end{bibeltext}
Das heisst für Menschen die jemandem andern ein Leid zu fügen sollen direkt aus der mitte der Bevölkerung entfernt werden. Was machen wir? Es wird geschaut was ist das für eine Peron. Ein armer Penner ist ein Monster und geht in den Knast. Ein reicher kann sich frei kaufen. Ein Priester und wichtige berühmte Personen wird versetzt und oder alles unter den Tisch gekehrt. Das ist unsere Ethik.\\ 
Manchmal habe ich das Gefühl, dass unsere Ethik nur so lange gilt wie andere etwas in unseren Augen schlechtes machen.\\
Ok, gehen wir zurück zu den Freuden die der \textsc{Herr} Israel mit seinen Gesetzen Israel geben möchte. Eigentlich hat er an alles gedacht. Eigentlich ist auch alles nicht so schwierig. Wenn vorher in Mose 3 die ganzen Opfer erklärt wurden, so sieht alles unglaublich kompliziert aus. Aber jetzt in Mose 5 scheint es, dass diese Opfergaben für den gewöhnlichen Mann nicht priorität hat. Ein Levit ist ein Priester der nichts besitzt und darum von den Gaben der Menschen lebt. Der Levit macht die Arbeit vom Volk für den \textsc{Herrn} wie unsere Priester. Die Leviten brauchten das Wissen wie man die Opfer dem \textsc{Herrn} bringt.\\
Da gibs ein intressanter Text den ich noch nicht so ganz verstehe:
\begin{bibeltext}{Sch2}{5Mos}{14:22-27}
    \textsuperscript{22} Du sollst allen \textbf{Ertrag deiner Saat getreu verzehnten}, was auf dem Feld wächst, Jahr für Jahr. \textsuperscript{23} Und du \textbf{sollst essen} vor dem \textsc{Herrn}, deinem Gott, an dem Ort, den er erwählen wird, um seinen Namen dort wohnwn zu lassen, den Zehnten deines Korns, deines Mosts, deines Öls und die Erstgeborenen von deinen Rindern und Schafen, damit du lernst, den \textsc{Herrn}, deinen Gott allezeit zu fürchten.\\
    \textsuperscript{24} Wenn dir aber der Weg zu weit ist und du es nicht hintragen kannst, weil der Ort, den der \textsc{Herr}, dein Gott, erwählen wird, um seinen Namen dorthin zu setzen, dir zu fern ist; wenn [nun] der \textsc{Herr}, dein Gott, dich segnet, \textsuperscript{25} so verkaufe es und \textbf{binde das Geld in deiner Hand zusammen} und geh an den Ort den der \textsc{Herr}, dein Gott, erwählen wird. \textsuperscript{26} \textbf{Und gib das Geld für das aus, was irgend dein Herz begehrt,} es sei für Rinder, Schafe, Wein, starkes Getränk oder was sonst deine Seele wünscht, und iss dort vor dem \textsc{Herrn}, deinem Gott, sei fröhlich, du und dein Haus. \textsuperscript{27} Den Leviten aber, der in deinen Toren ist, sollst du nicht im Stich lassen; denn er hat weder Teil noch Erbe mit dir.
\end{bibeltext}
Hier heisst es, dass der Zehnte selber gegegessen werden kann, aber mit dem Levit geteilt werden soll. Es ist immer wieder Intressant, dass der \textsc{Herr} doch immer wieder mit \textit{soll} redet und nicht mit \textit{du musst}. Das heisst nicht, dass wenn ich mache was ich will, keine konsequenzen tragen muss, sondern ich kann mich auch anders entscheiden. Aber ich muss dann auch die Verantwortung für meine Entscheidung übernehmen. Aber wenn es dann soweit ist, ist alles andere Schuld, nur nicht meine eigenen Fehler.\\
Wie das jetzt aber mit dem Zehnten funktioniert, habe ich noch nicht so ganz verstanden. Was ich aber verstanden habe ist, dass man den Zehnten selber essen soll, aber auch die Leviten dazu einladen soll. Nach dem Motto fröhlich sein und feiern.\\
Was machen wir ach so gläubigen Heute? Wir versuchen verbissen, Gott etwas gutes zu tun. \flqq Der Zehnte müssen wir geben weil Gott es so will.\frqq{} wir sagen es nicht, aber wenn dann das Geld vom Konto geht denken wir es. \flqq Ist ja für ein gutes Werk.\frqq{} Wir sollen aber auch den Zehnten von allem auf die Seite legen und dann mit anderen zusammen feiern und fröhlich sein.\\


 
