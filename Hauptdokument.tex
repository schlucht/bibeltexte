\documentclass[a4paper, 12pt]{scrartcl}
\usepackage[utf8]{inputenc}
\usepackage[T1]{fontenc}
\usepackage{lmodern}
\usepackage{ngerman}
\usepackage{bibleref}
\usepackage{bibleref-german}
\usepackage{blindtext}

\begin{document}
	\verb=\=textbf\{\textbf{Fettgedruckte Zeichen}\}
	\verb=\=textit\{\textit{Kursive Zeichen}\}
	
	\bibleverse{Rev}(13:)\\
	\bibleverse{Matt}(2:1-3)\\
	\newcommand{\bibtit}[1]{\textbf{\large#1}\\}
	\newcommand{\bib}[1]{%
		\ifthenelse{\equal{#1}{LuN}}{Neue Luther}{}%
		\ifthenelse{\equal{#1}{Sch2000}}{Schlachter 2000}{}%
	}
	
	\newenvironment{bibeltext}[3]{%
		\quote \begin{itshape}
		\begin{scriptsize}
			(\bib{#1})
		\end{scriptsize}
		\biblerefformat{lang}
		\textbf{\bibleverse{#2}(#3): }
	}{%	
		\textbf{2}
		\end{itshape}
		\endquote		
	}
	
	\begin{bibeltext}{LuN}{Matt}{1:1-4}
		\blindtext
	\end{bibeltext}
	
	\begin{flushright}
	
	Die ist ein längerer Text mit einem Zeilenumbruch. Ich fange jetzt an ein Buch zu schreiben und möchte dies gerne endlich mal richtig machen. \\
	Da das ganze nicht immer so einfach ist, fange ich jetzt einfach mal an und schreibe ganz einfach mal was. 1. Versuch beim schreiben eines Textes funktioniert es doch schon mal ganz gut oder?\\[.5cm]
	Nach längerer Zeit habe ich mich wieder hingesetzt und wieder angefangen zu schreiben.\\
	Das ganze wird aber schon ganz schön lang und erfordert ganz sicher auch seine Zeit um das ganze in die richtige Perspektive zu kriegen.\\
	\begin{flushleft}
	Linksaugerichteter Text. Dieser Text sollte sich links ausrichten, aber keine Silbentrennung beinhalten. Dieser Text wird einfach auf Links ausgerichtet und so ein neues Schriftbild generiert. Werden das sollte vielleicht eines Tages einem Helfen und die Sache besser ausrichten zu können.
	\end{flushleft}
	\footnote{Neuer Text}
	Dieser Text wird Rechts ausgerichtet und so auf der rechten Seite aufgereit werden. Der Text wird wieder nicht umgebrochen, das heisst es findet keine Silbentrennung statt, sondern wird einfach nur das lange Wort in die neue Zeile gelegt.\marginpar{rechts} Dies sollte eigentlich dann auf der Rechten Seite ausgerichtet werden. Mal schauen wie das ganze ausgerichtet wird.
	\end{flushright}
	\footnote{2. Neuer Text}
	\textsc{Hoffnung für Alle}
	\begin{quote}
	| Am Anfang erschuf Gott der Himmel und die Erde. \\
	| Am 2. Tag erschuf
	\end{quote}
	\begin{itemize}
	\item Dies ist eine Aufzählung mit mehreren Einträgen die auch länger mal sind. Neben dem Versuch zu lernen versuch ich auch noch eine Internetseite zu gestalten. 
	\item versuch mit \texttt{Typescript} und Angular 4. Ist nicht so einfach das ganze, aber es funktioniert schon mal nicht so schlecht. 
	\item CSS ist wesentlich schwieriger und Aufwendiger.
	\end{itemize}
	
	
	
\end{document}
