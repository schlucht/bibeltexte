\documentclass[a4paper, 12pt, twoside, titlepage,headsepline]{scrartcl}

\usepackage[utf8]{inputenc}
\usepackage[T1]{fontenc}
\usepackage{lmodern}
\usepackage{ngerman}
\usepackage{bibleref}
\usepackage{bibleref-german}
\usepackage{blindtext}
\usepackage{graphicx}
\usepackage{geometry}
\usepackage[babel, german=swiss]{csquotes}

%%%%%%%%%%%%%%%%%%%%%%Link Formatierung%%%%%%%%%%%%%%%%%%%%%%%%%%%
\usepackage[colorlinks = true,
linkcolor = black,
urlcolor  = blue,
citecolor = black,
anchorcolor = blue]{hyperref}

%%%%%%%%%%%%%%%%%%%%%Bibliothek%%%%%%%%%%%%%%%%%%%%%%%%%%%%%%%%%%%
\usepackage[backend=bibtex,style=verbose,backref=true,]{biblatex}
\setlength{\bibitemsep}{1em}     % Abstand zwischen den Literaturangaben
\setlength{\bibhang}{2em}        % Einzug nach jeweils erster Zeile

\addbibresource{literatur.bib}

%%%%%%%%%%%%%%%%%%%%%%%%%%%%%%%%%%%%%%%%%%%%%%%%%%%%%%%%%%%%%%%%%%
%
% Makro zur Namensvervollstädigung
% Parameter: Kürzel --- Bibel
%			LuN			Neue Luther Übersetzung
%			Sch2		Schlachter 2000
%			Elb			Elbfelder
%%%%%%%%%%%%%%%%%%%%%%%%%%%%%%%%%%%%%%%%%%%%%%%%%%%%%%%%%%%%%%%%%%%
	\newcommand{\bib}[1]{%
	\ifthenelse{\equal{#1}{LuN}}{Neue Luther}{%
		\ifthenelse{\equal{#1}{Sch2}}{Schlachter 2000}{%
			\ifthenelse{\equal{#1}{HFA}}{Hoffnung für Alle}{%
				\ifthenelse{\equal{#1}{ELB}}{Elbfelder}{#1}%
			}%
		}%
	}%
}
%%%%%%%%%%%%%%%%%%%%%%%%%%%%%%%%%%%%%%%%%%%%%%%%%%%%%%%%%%%%%%%%%%
%
% Makro für Bibelzitate
% 
% Beispiel: 
%		\begin{bibeltext}{ELB}{Matt}{1:1-4}
%			Ich bin der zitierte Bibeltext.
%		\end{bibeltext}
%
%%%%%%%%%%%%%%%%%%%%%%%%%%%%%%%%%%%%%%%%%%%%%%%%%%%%%%%%%%%%%%%%%%
\newcommand{\bibtit}[1]{\large\bib{#1}}
\newcommand{\tmpbib}{Hallo}
\newenvironment{bibeltext}[3]{%
	\quote \begin{itshape}
		\begin{scriptsize}
			(\bib{#1})	
		\end{scriptsize}	
		\biblerefformat{lang}
		\renewcommand{\tmpbib}{\textbf{\bibleverse{#2}(#3)}}		
	}{%	
		\tmpbib
	\end{itshape}
	\endquote		
}
%%%%%%%%%%%%% Titelseite %%%%%%%%%%%%%%%%%%%%%%%%%%%%%%%%%%%%%%%%%%%%%
\title{Freikirchen, Katholische Kirchen, Evangelische Kirchen}
\author{Schmid Lothar}
\date{\today}
\pagestyle{headings}

%%%%%%%%%%%%%%%%% Beginn Dokument %%%%%%%%%%%%%%%%%%%%%%%%%%%%%%%%%%%%
\begin{document}

\maketitle[-1]

\include{inhaltsverzeichnis}
\section{Vorwort}

Wenn sich ein bisschen mit den Kirchen, Religionen und Glauben beschäftigt, kommt man nicht um den Begriff Ökumene vorbei. Es gibt Kirchen die Ökumene als die Zukunft und das einzige heilbringende darstellen und andere wiederum die Ökumene verteufeln.\\
Ich weiss nicht ob es wirklich gut ist oder nicht. Darum versuche ich in diesem Schreiben meine Gedanken zu Papier zubringen und schauen was denn die Bibel dazu sagt. In erster Linie tönt es gut. Gemeinschaft mit anderen, anderen Gemeinschaften Respekt entgegen bringen...\\
Ist das wirklich so? Welche Religion will jetzt welche aufnehmen? Geht es um religiöse Fusionen oder um Mitgliederbewerbung. \\ 
Die Welt des Internet ist voll davon. Auf YouTube gibt es hunderte von Predigten welch sich um Ökumene befassen. Ich bin weder Psychologe noch Theologe sondern einfach nur ein Gläubiger Christ den das ganze interessiert und gerne seine Gedanken zu Papier bringen möchte.\\
Ich möchte das gerne mit dem Geist Gottes zusammen machen. So dass er mich leitet und mich unterstützt.\\\\

In dem Sinne werde ich meine Gedanken zu Papier bringen.






\include{bibliothek}

%% Zu letzt einbinden für die Textreferenzen %%

\section*{Literaturverzeichnis} 
\addcontentsline{toc}{section}{Literaturverzeichnis} 
\printbibliography[title={Literaturverzeichnis}] 
%\printbibliography[heading=subbibliography]

\end{document}
