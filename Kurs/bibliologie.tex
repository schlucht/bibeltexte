\chapter{Bibliologie}
\section{Das Wort Gottes}
\subsection {Woher kommt das Wort Bibel}
Das Wort Bibel heisst, einfach Buch. Es wird von dem griechischen Wort \textit{biblios} abgeleitet. Ihre aussergewöhnliche Wesensart beruht auf der Tatsache, dass sie die Wahrheit das Wort Gottes ist, obwohl sie von menschlichen Autoren geschrieben wurde.\\

\subsection{Arten der Beweisführung:}
\begin{enumerate}
	\item die internen Beweise, die in der Bibel selbst zu finden Tatsachen.
	\item die externen Beweise, das Wesen der in der hl.\ Schrift gegeben Tatsachen.
\end{enumerate}

\subsection{5 Beispiele für die interne Beweisführung}
Fünf Beispiele der internen Beweisführung im alten Testament:
\renewcommand{\labelenumi}{\Roman{enumi}}
\begin{enumerate}
	\item  
	\begin{bibeltext}{Sch2}{Deut}{6:6-7}
		Und diese Worte die ich dir heute gebiete, sollst du auf dem Herzen tragen, und du sollst sie deinen Kindern einschärfen und davon reden, wenn du in deinem Haus sitzt oder auf dem Weg gehst, wenn du dich niederlegst und wenn du aufstehst.
	\end{bibeltext}
	\item 
	\begin{bibeltext}{Sch2}{Jos}{1:8}
		Lass dieses Buch des Gesetzes nicht von deinem Mund weiche, sondern forsche darin Tag und Nacht damit du darauf achtest, alles zu befolgen, was darin geschrieben steht; denn dann wirst du Gelingen haben auf deinen Wegen, und wirst du weise handeln!
	\end{bibeltext}
	\item 
	\begin{bibeltext}{Sch2}{Jos}{8:32-35}
		Und er schrieb dort auf die Steine eine Abschrift des Gesetzes Moses, das er in Gegenwart der Kinder Israels geschrieben hatten. Und ganz Israel samt seinen Ältesten und Vorsteher und Richtern stand zu beiden Seiten der Lade, den Priestern und den Leviten gegenüber, welche die Bundeslade des Herrn trugen, die Fremdlinge wie auch die Einheimischen; die eine Hälfte gegenüber dem Berg Garazim und die andere Hälfte gegenüber dem Berg Ebal, wie Mose der Knecht des \herr n, zuvor geboten hatte, das Volk Israel zu segnen, Danach las er alle Worte des Gesetzes, den Segen und Fluch, alles, wie es im Buch des Gesetzes geschrieben steht.
	\end{bibeltext}
\item 
	\begin{bibeltext}{Sch2}{Ps}{1:2}
		sondern seine Lust hat am Gesetz des \herr n und über sein Gesetz nachsinnt Tag und Nacht.
	\end{bibeltext}
\item 
	\begin{bibeltext}{Sch2}{Spr}{30:5-6}
		Alle Reden Gottes sind geläutert;\\
		er ist ein Schild denen, die ihm vertrauen.\\
		Tue nichts zu seinem Worten hinzu,\\
		damit er dich nicht bestraft und du als Lügner dastehst.\\
	\end{bibeltext}
\end{enumerate}
\subsection{5 Beispiele für die Externe Beweisführung im NT}
\renewcommand{\labelenumi}{\Roman{enumi}}
\begin{enumerate}
	\item  
	\begin{bibeltext}{Sch2}{Mk}{13:31}
	Himmel und Erde werden vergehen, aber meine Worte werden nicht vergehen.
	\end{bibeltext}
	\item 
	\begin{bibeltext}{Sch2}{Lk}{16:17}
		Es ist aber leichter, dass Himmel und Erde vergehen, als dass ein einziges Strichlein des Gesetzes falle.
	\end{bibeltext}
	\item 
	\begin{bibeltext}{Sch2}{Rom}{10:17}
		Demnach kommt der Glaube aus der Verkündigung, dei Verkündigung aber durch Gottes Wort.
	\end{bibeltext}
	\item 
	\begin{bibeltext}{Sch2}{Kol}{3:16}
		Lasst das Wort des Herrn reichlich in euch wohnen in aller Weisheit; lehrt und ermahnt einander und singt mit Psalmen und Lobgesängen und geistlichen Liedern dem \herr n lieblich in eurem Herzen.
	\end{bibeltext}
	\item 
	\begin{bibeltext}{Sch2}{Offb}{22:18}
		Fürwahr ich bezeuge jedem, der die Worte der Weisssagung dieses Buches hört: Wenn jemand etwas zu diesen Dingen hinzufügt, so wird Gott ihm die Plagen zufügen, von denen in diesem Buch geschrieben stehen.
	\end{bibeltext}
\end{enumerate}
\section{Die Bibel von Gott inspiriert}
\subsection{Was ist mit der Inspiration gemeint?}
Die Inspiration ist die Lehre, dass Gott menschliche Schreiber einsetzt, die sein Wort aufschreiben, ohne ihnen ihre literarischen, persönlichen Interessen und eigene Individualität zu zerstören. Es wurden zwar von Gott menschliche Schreiber eingesetzt, aber die behielten ihren eigen Stil.
\subsection{Inwieweit ist die Bibel inspiriert}
Die Bibel ist vollständig von Gott inspiriert. Auch wenn einzelne Bibelstellen sich in ihrer Eigenart hervorheben, sind diese inspiriert. Gott führte die Männer in allen Einzelheiten. In ihren Gedanken, Hoffnungen, und Ängsten.
\subsection{Was ist mit wörtlicher, vollständiger Inspiration gemeint?}
Gott führte in allen Einzelheiten die Schreiber so, dass sie alles das schrieben was er meinte. Auch wenn einzelne Stellen erheblich von einander unterscheiden sind sie immer noch Gottes Wort.
\section{Ihr Thema und ihr Zweck}
\subsection{Welche Beweise gibt es dafür, dass Christus an der Schöpfung mit gewirkt hat?}
\begin{bibeltext}{Schl2}{Joh}{1:3}
	Alles ist durch denselben entstanden; und ohne dasselbe ist auch nicht eines entstanden, was entstanden ist.
\end{bibeltext}
" duch denselben " steht für Jesus. Dies kommt in den Versen 1 und 2 zur Geltung. Das heisst dass ohne den Jesus nichts entstanden ist. Was aber nicht bedeutet das Gott und der hl.\ Geist nicht mit im Spiel war.
\subsection{Inwiefern ist Christus der oberste Herrscher der Welt, und wie zeigt sich das?}
	Kurz weil er der Schöpfer ist. Auch wenn die hl.\ Schrift dem Vater die höchste Herrschaft zu spricht, ist es doch klar erkennbar Seine Absicht, dass Christus über die Welt herrschen soll. Der Tag wird kommen an dem Christus über alles sein wird, der Tag an dem die Sünder gerichtet werden und Seine Herrschaft offenbart werden wird.
	\begin{bibeltext}{Schl2}{Offb}{19:15-16}
		Und aus seinem Mund geht ein scharfes Schwert hervor, damit er die Heidenvölker mit ihm schlage, und er wird sie mit eisernem Stab weiden; und er tritt die Weinkelter des Grimmes und des Zornes Gottes, des Allmächtigen. Und er trägt an seinem Gewand und an seiner Hüfte den Namen geschrieben: \textquote{König der Könige und Herr der Herren.}
	\end{bibeltext}
\section{Eine göttliche Offenbarung}
\subsection{Warum will Gott sich uns offenbaren}
Es ist nur natürlich, dass vernunftbegabte Lebewesen versuchen sollten, etwas über seinen Schöpfer zu erfahren. Es ist auch vernünftig zu erwarten, dass ein Schöpfer der begabte und vernünftige Menschen erschaffen hat, Kontakt mit ihnen aufnehmen möchte. Gott gebraucht drei Hauptwege um sich ihm zu Offenbaren.
\begin{enumerate}
	\item Die Offenbarung Gottes in der Schöpfung.
	\item Die Offenbarung in Christus
	\item Die Offenbarung im geschriebenen Wort
\end{enumerate}
\subsection{Welchen Umfang und Grenzen hat die Offenbarung in der Natur}
Die natürliche Welt als Werk Gottes enthüllt, dass Gott ein Gott mit unendlicher Macht und Weisheit ist und die materielle Welt zu einem bestimmten Zweck erschaffen hat. In der Natur sind weder die Liebe noch seine Heiligkeit deutlich sichtbar. Sie offenbart keinen Weg der Erlösung, durch den die Sünder sich mit einem heiligen Gott versöhnen können.
\subsection{Inwieweit ist Christus die Offenbarung Gottes?}
In Christus werden nicht nur die Macht und Weisheit Gottes offenbart, sondern auch seine Liebe, Güte, Heiligkeit und Gnade. Christus erklärt:
\begin{bibeltext}{Elb}{Joh}{14:9}
	Jesus spricht zu ihm: \textquote{So lange Zeit bin ich bei euch, und du hast mich nicht erkannt, Phillipus? Wer mich gesehen hat, hat den Vater gesehen, und wie sagst du: Zeige uns den Vater?}
\end{bibeltext} 
Jesus kam also in die Welt um uns Gott zu zeigen und uns zu offenbaren.
\subsection{Warum ist die Schrift wichtig um Gott vollständig zu offenbaren?}
Das geschriebene Wort ist in der Lage Gott ausführlicher zu offenbaren als in der Person und den Werken Christi. Die Bibel enthüllt uns Gottes Plan für Israel, die Nationen und die Gemeinde und die Absichten die Gott hat. Die geschriebene Offenbarung ist all umfassend. Die Bibel kann folglich als Vervollständigung der beabsichtigten göttlichen Offenbarung gesehen werden, wie sie teilweise in der Natur gegeben ist, ausführlicher in Christus und vollständig im geschriebenen Wort. 
\section{Der dreieinige Gott}
\subsection{Wie können wir den allgemeinen Glauben an die Existenz Gottes erklären?}
Der Mensch scheint sich intuitiv, allein schon aufgrund seiner religiösen Natur, nach einer Art höheren Wesen auszustrecken. Die Beweise für die Existien Gottes die in der Schöpfung vorliegen, sind so eindeutig, dass ihre Leugnungen der Grund für die Verdammung der heidnischen Welt ist, die das Evangelium nicht vernommen hat.

Obwohl die Welt im Allgemeinen die Schriftoffenbarung nicht kennt, sind doch einige Beriffe von Gott in das Denken der gesamten Welt eingedrungen, sodass der Glaube an einen Art höhreres Wesen selbst unter Menschen allgemein anerkannt wird die, nicht direkt mit der Schrift in Berährung gekommen sind.\\
Verschiende Gedankensysteme:
\begin{itemize}
	\item \textbf{der Polytheismus}, der Glaube an viel Götter
	\item \textbf{der Hylozoismus}, der das in der gesamten Schöpfungzu findende Lebensprinzip mit Gott selbst identifiziert
	\item \textbf{der Materialismus}, der die Meinung vertritt, dass die Materie selbständig, aufgrund von Naturgesetzen funktioniert und dass kein Gott hierfür erforderlich ist
	\item \textbf{der Pantheismus}, der behauptet, dass Gott unpersönlich und mit der Natur identisch, also immanent statt transzendent sei.
\end{itemize}
Von diesen Gottesbegriffen existieren wiederum viele Variationen.\\
Wenn man für die Existenz Gottes auf die Grundlage der Schöpfungstatsachen argumentiert, kann viel Argumentationsklassen unterscheiden.
\begin{enumerate}
	\item Das \textbf{ontologische Argument}, Gott existiert, weil die Menschen an ihn glauben.
	\item Das \textbf{kosmologische Argument}, dass jede Wirkung eine Ursache hat und deshalb die Ursache einen Schöpfer haben muss.
	\item Das \textbf{theologische Argument}, dass jeder Entwurf einen Konstrukteur haben muss und dass die gesammte Schöpfung komplex konstruiert und miteinnander verknüpft ist.
	\item \textbf{Das anthropologische Argument}, dass sich die Existenz der Menschen ohne Schöpfer nicht erklären lassen kann. 
\end{enumerate}
Obwohl diese Argumente für die Existenz Gottes eine beträchtliche Stichhaltigkeit aufweisen, so ist doch erst in der Bibel die vollständige Offenbarung Gottes gegeben.
\subsection{Warum ist der Atheismus unvernünftig?}
Man kann das Universum nicht wegdenken. Der Atheismus beruht darauf, dass alles von alleine entstanden ist. Eine Leugnung jeder rationalen Erklärung des Universum.
\section{Gott der Vater}
\subsection{Wie werden die Werke des Vaters, des Sohnes und des Heiligen Geistes im Neuen Testament einander gegenübergestellt?}
Der Vater wird dargestellt als der, der erwählt, liebt und schenkt. Der Sohn wird dargestellt als der, der leidet, erlöst und das Universum erhält. Der Heilige Geist wird dargestellt als der, der von Neuem zeugt, der in uns wohnt, uns tauft, kräftigt und heiligt.
\subsection{Was sind die vier verschiedenen Aspekte der Vaterschaft Gottes?}
\begin{enumerate}
	\item Gott als der Vater der gesammten Schöpfung
	\item Gott der Vater auf Grund enger Gemscheinschaft
	\item Gott als der Vater unserers Herren Jesus Christus
	\item Gott als der Vater all derer, die an Jesus Christus als Herrn und Retter glauben.
\end{enumerate}
\section{Gott der Sohn: Seine Gottheit und Ewigkeit}