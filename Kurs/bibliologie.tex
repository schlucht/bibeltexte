\subsection{Das Wort Gottes}
\subsubsection {Woher kommt das Wort Bibel}
Das Wort Bibel heisst, einfach Buch. Es wird von dem griechischen Wort \textit{biblios} abgeleitet. Ihre aussergewöhnliche Wesensart beruht auf der Tatsache, dass sie die Wahrheit das Wort Gottes ist, obwohl sie von menschlichen Autoren geschrieben wurde.\\

\subsubsection{Arten der Beweisführung:}
\begin{enumerate}
	\item die internen Beweise, die in der Bibel selbst zu finden Tatsachen.
	\item die externen Beweise, das Wesen der in der hl. Schrift gegeben Tatsachen.
\end{enumerate}

\subsubsection{5 Beispiele für die interne Beweisführung}
Fünf Beispiele der internen Beweisführung im alten Testament:
\renewcommand{\labelenumi}{\Roman{enumi}}
\begin{enumerate}
	\item  
	\begin{bibeltext}{Sch2}{Deut}{6:6-7}
		Und diese Worte die ich dir heute gebiete, sollst du auf dem Herzen tragen, und du sollst sie deinen Kindern einschärfen und davon reden, wenn du in deinem Haus sitzt oder auf dem Weg gehst, wenn du dich niederlegst und wenn du aufstehst.
	\end{bibeltext}
	\item 
	\begin{bibeltext}{Sch2}{Jos}{1:8}
		Lass dieses Buch des Gesetzes nicht von deinem Mund weiche, sondern forsche darin Tag und Nacht damit du darauf achtest, alles zu befolgen, was darin geschrieben steht; denn dann wirst du Gelingen haben auf deinen Wegen, und wirst du weise handeln!
	\end{bibeltext}
	\item 
	\begin{bibeltext}{Sch2}{Jos}{8:32-35}
		Und er schrieb dort auf die Steine eine Abschrift des Gesetzes Moses, das er in Gegenwart der Kinder Israels geschrieben hatten. Und ganz Israel samt seinen Ältesten und Vorsteher und Richtern stand zu beiden Seiten der Lade, den Priestern und den Leviten gegenüber, welche die Bundeslade des Herrn trugen, die Fremdlinge wie auch die Einheimischen; die eine Hälfte gegenüber dem Berg Garazim und die andere Hälfte gegenüber dem Berg Ebal, wie Mose der Knecht des \herr n, zuvor geboten hatte, das Volk Israel zu segnen, Danach las er alle Worte des Gesetzes, den Segen und Fluch, alles, wie es im Buch des Gesetzes geschrieben steht.
	\end{bibeltext}
\item 
	\begin{bibeltext}{Sch2}{Ps}{1:2}
		sondern sein Lust hat am Gesetz des \herr n und über sein Gesetz nachsinnt Tag und Nacht.
	\end{bibeltext}
\item 
	\begin{bibeltext}{Sch2}{Spr}{30:5-6}
		Alle Reden Gottes sind geläutert;\\
		er ist ein Schild denen, die ihm vertrauen.\\
		Tue nichts zu seinem Worten hinzu,\\
		damit er dich nicht bestraft und du als Lügner dastehst.\\
	\end{bibeltext}
\end{enumerate}
\subsubsection{5 Beispiele für die Externe Beweisführung im NT}
\renewcommand{\labelenumi}{\Roman{enumi}}
\begin{enumerate}
	\item  
	\begin{bibeltext}{Sch2}{Mk}{13:31}
	Himmel und Erde werden vergehen, aber meine Worte werden nicht vergehen.
	\end{bibeltext}
	\item 
	\begin{bibeltext}{Sch2}{Lk}{16:17}
		Es ist aber leichter, dass Himmel und Erde vergehen, als dass ein einziges Strichlein des Gesetzes falle.
	\end{bibeltext}
	\item 
	\begin{bibeltext}{Sch2}{Rom}{10:17}
		Demnach kommt der Glaube aus der Verkündigung, dei Verkündigung aber durch Gottes Wort.
	\end{bibeltext}
	\item 
	\begin{bibeltext}{Sch2}{Kol}{3:16}
		Lasst das Wort des Herrn reichlich in euch wohnen in aller Weisheit; lehrt und ermahnt einander und singt mit Psalmen und Lobgesängen und geistlichen Liedern dem \herr n lieblich in eurem Herzen.
	\end{bibeltext}
	\item 
	\begin{bibeltext}{Sch2}{Offb}{22:18}
		Fürwahr ich bezeuge jedem, der die Worte der Weisssagung dieses Buches hört: Wenn jemand etwas zu diesen Dingen hinzufügt, so wird Gott ihm die Plagen zufügen, von denen in diesem Buch geschrieben stehen.
	\end{bibeltext}
\end{enumerate}
\subsection{Die Bibel von Gott inspiriert}
\subsubsection{Was ist mit der Inspiration gemeint?}
Die Inspiration ist die Lehre, dass Gott menschliche Schreiber einsetzt, die sein Wort aufschreiben, ohne ihnen ihre literarischen, persönlichen Interessen und eigene Individualität zu zerstören. Es wurden zwar von Gott menschliche Schreiber eingesetzt, aber die behielten ihren eigen Stil.
\subsubsection{Inwieweit ist die Bibel inspiriert}
Die Bibel ist vollständig von Gott inspiriert. Auch wenn einzelne Bibelstellen sich in ihrer Eigenart hervorheben, sind diese inspiriert. Gott führte die Männer in allen einzelheiten. In ihren Gedanken, Hoffnungen, und Ängsten.
\subsubsection{Was ist mit wörtlicher, vollständiger Inspiration gemeint?}
Gott führte in allen Einzelheiten die Schreiber so, dass sie alles das schrieben was er meinte. Auch wenn einzelne Stellen erheblich voneinnander unterscheiden sind sie immer noch Gottes Wort.
\subsection{Ihr Thema und ihr Zweck}
\subsection{Eine göttliche Offenbahrung}



