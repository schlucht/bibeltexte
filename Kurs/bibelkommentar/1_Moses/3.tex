
\subsubsection{1. Mose 3 (Genesis 3)}
In diesem Kapitel wird der Sündenfall des Menschen beschrieben. Die Geschehnisse in diesem Kapitel sind der Grund wieso die Bibel Überhaupt geschrieben werden musste. Nach dem Sündenfall des Menschen, musste Gott den ganzen Heilsplan in Bewegung setzten, um die Menschen wieder in seine Nähe zu ziehen.

Es ist erstaunlich, dass schon diese zwei Menschen schon daran interessiert waren, Macht zu besitzen. Eigentlich ging es doch ihnen gut im Paradies. Hatten alles, aber trotzdem wollten sie mehr. Sie wollten so sein wie Gott. Die Schlange hat es ihnen versprochen. Sie müssen nur von diesem Baum essen.
\begin{bibeltext}{Sch2}{Gen}{3:1}
	Aber die Schlange war listiger als alle Tiere des Feldes, die Gott der Herr gemacht hatte; und sie sprach zur Frau: Sollte Gott wirklich gesagt haben, dass ihr von \textbf{keinem} Baum im Garten essen dürft?
\end{bibeltext}
Das hat doch die Schlange geschickt eingefädelt oder? Sie macht der Frau den Mund richtig wässrig. Kann doch nicht so schlimm sein. Gott will euch doch etwas vorenthalten. \textquote{Ihr werdet sein wie Gott... }. Wollen wir das nicht auch heute noch? Sein wie Gott? Flüstert der Satan uns nicht immer noch ins Ohr, \textquote{Hat er euch wirklich diese strengen Gebote gegeben? Hat Gott verboten Spass zu haben? Er ist doch ein Spielverderber. Hört nicht auf ihn. Ihr seid viel schlauer als er. Mit der Wissenschaft seid ihr die Götter}.
Nachdem beide von dem tollen Baum gegessen haben geschieht es. \textquote{\textbf{Da erkannten sie dass sie nackt waren...}}. Ende letztes Kapitel heisst es: \textquote{und sie waren nackt und schämten sich nicht}. Das ist die Erkenntnis von Gut und Böse. Jetzt sieht der Mensch das Böse vom anderen. Sie schämten sich vor einander, sie konnten sich gegenseitig nicht mehr in die Augen blicken.\\
Als dann Gott sie suchte versteckten sie sich im Garten. Das Raus reden ist typisch für uns Menschen. Der Mann schiebt die Verantwortung auf die Frau und auf Gott. \textquote{Du hast mir die Frau gegeben}. Die schiebt die Schuld auf die Schlange. Also genau so wie wir uns noch Heute aus der Verantwortung winden wollen.

Diese Aktion hatte weitreichende Auswirkungen die wir noch heute spüren. Es kam der Tod, der Schmerz in die Welt. Die Auswirkungen waren so schlimm, dass Gottes Plan schon zu dieser Zeit festgelegt war, seinen Sohn für die Erlösung von uns Menschen zu schicken. Die Menschheit musste aber zuerst vorbereitet werden. Hier tauchen die ersten Hinweise auf das Opfer von seinem Sohn. So sagt Gott zu der Schlange:
\begin{bibeltext}{Sch2}{Gen}{3:15}
	Und ich will Feindschaft setzten zwischen dir und der Frau, zwischen deinem Samen und ihrem Samen: Er wird dir den Kopf zertreten, und du wirst ihn in die Ferse stechen.
\end{bibeltext}
Hier wird gezeigt wie Jesus den Satan besiegt, \textquote{er wird ihm den Kopf zertreten, und du wirst ihm in die Ferse stechen} meint die Kreuzigung Jesus. Im weiteren Verlauf der Bibel werden die Hinweise auf Jesus immer konkreter.
Auch die beiden Menschen bekamen ihre Strafe, mit der wir noch jetzt leben und zu kämpfen haben. Geburtsschmerzen, Krankheit auch die strenge Arbeit zum überleben, hat von an ihren Anfang.

Adam\footnote{Hebr. Adama: Erdboden. Dient als Eigennamen für den Mensch} nannte seine Frau Eva\footnote{Hebr. Chawa: Leben} und Gott stattete die beiden mit Kleider aus Fellen aus. Die Auslegung sagt, dass die Felle das erste Blutopfer für die begangene Sünde des Menschen war. Dar letzte Blutopfer hat Jesus der Christus für uns am Kreuz in Golgotha vollbracht.