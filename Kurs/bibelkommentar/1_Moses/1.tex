
\subsubsection{1. Mose 1 (Genesis 1)}
Das erste Kapitel gibt einen Überblick wie Gott das Universum (Verse: 1 - 5), die Wasser (Verse 6 - 10), die Pflanzen (Verse 11 - 13), die Tage ( Verse 14 - 19), die Tiere (Verse 20 -25) und die Menschen (Verse 26 - 27) geschaffen hat. Die Erschaffung des Menschen ist im 1 Kapitel relativ kurz gehalten. Diese wird dann im Kapitel 2 ausführlicher behandelt.

Jede Schöpfung geschieht an einem Tag. Am Ende des Tages heisst es immer: \textquote{
Und es wurde Abend, und es wurde Morgen der fünfte Tag.} Das heisst der Tag hatte einen normalen Ablauf. Am Morgen aufstehen mit der Arbeit beginnen und am Abend die Arbeit niederlegen und ab in den Feierabend.

Es wird auch aufgezeigt, dass jeden Abend Gott sein Werk betrachtete und mit seiner Arbeit zu frieden war.

Ich finde das wir aus diesem Kapitel lernen können wie wir unsere Tage gestalten sollten. Am Tag die Arbeit verrichten und nach getaner Arbeit zufrieden auf diese zurückblicken. Es ist auch wichtig, dass wir wirklich in den Feierabend gehen und die Arbeit bis am Morgen wieder ruhen lassen.

Gott hat nicht alles an einem Tag erschaffen um der Rest der Woche frei zu haben. Es ist sehr wichtig, dass wir mit unserem Tag zufrieden sind. Als ehemaliger Bäcker weiss ich, dass die Nachtarbeit sehr anstrengend ist. Da ist es sehr wichtig, dass man sich die Erholung am Tag holt. Dies ist natürlich schwieriger, weil es hell und lauter ist.