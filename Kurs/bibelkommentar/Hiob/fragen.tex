\subsection{Übersichtsfragen zu Hiob}
\begin{enumerate}
    \item \textbf{Wie lauten einige Charaktereigenschaften Hiobs}\\
    Untadelig, rechtschaffen, gottesfürchtig, geistlich gesinnt, keusch, hilfsbereit
    \item \textbf{Warum muss Hiob leiden?}\\
    Es tobt ein Kampf zwischen Gott und den Teufel. Der ist der Meinung, dass sobald es Hiob schlecht gehen wird dieser vom Glauben abfallen wird.
    \item \textbf{Wie lassen sich die ersten sechs Prüfungen Hiobs zusammenfassen?}\\
    Vernichtung seines gesamten Besitzes, Zerstörung seiner Gesundheit, Ratschlag seiner Ehefrau sich von Gott loszusagen und Selbstmord zu begehen.
     \item \textbf{Was führte zum Zerbruch von Hiob?}\\
     Die siebte Prüfung: Seine takt- und lieblosen, besserwisserischen und verurteilenden Freunde.
     \item \textbf{Was lernen wir aus dem Buch Hiob?}\\
     \begin{itemize}
     		\item Gott beanwortet nicht alle \flqq Warum-Fragen \frqq
     		\item Gott ist würdig, unabhängig von allen Lebensumstaänden angebetet zu werden
     \end{itemize}
     \item \textbf{Was bewirkt die grosse Wende?}\\
     Die Begegnung mit Gott, gefolgt von Busse und Beugung; das Gebet für seine Freunde war der Auslöser für vollständige Wiederherstellung.
     \item \textbf{Wo und wann fand die Geschichte Hiobs statt?}\\
     Im Land Uz, in Edom, was dem heutigen Jordanien entspricht, im 3. Jahrtausend vor Christus.
     
\end{enumerate}