\section{Übersichtsfragen zu 4. Mose}
\begin{enumerate}
    \item textbf{Warum heisst 4. Mose auch «In der Wüste»?}\\
    Gott hat sein Volk mit starker Hand erlöst. Aus ihnen ein Volk geformt. Sich ihnen und wunderbarer und herrlicher Weise offenbart. Doch Isreal reagiert immer wieder in Unglauben und in Auflehnung bis das \flqq bis das Fass überläuft \frqq, bis zum \flqq Point of no return \frqq.
    \begin{bibeltext}{Sch2}{Num}{14:11}
        Und der \herr sprach zu Mose: Wie lange noch will mich dieses Volk verachten? Und wie lange noch wollen, sie nicht an mich glauben, trotz aller Zeichen, die ich unter ihnen getan habe?
    \end{bibeltext}
    \begin{bibeltext}{Sch2}{Num}{14:22-23}
        Keiner der Männer die meine Herrlichkeit und meine Zeichen gesehen haben, die ich in Ägypten und in der Wüste getan habe, und die mich nun schon zehnmal versucht und meiner Stimme nicht gehorcht haben, soll das Land sehen, das ich ihren Vätern zugeschworen habe; ja keiner soll es sehen, der micht verachtet hat!
    \end{bibeltext}
    Entsprechent beginnt eine 40jährige Wüstenwanderung, voll Not und Elend, bis alle, Gottes Wege verwarfen, in der Wüsten starben und somit das Land der Verheissung nicht zu sehen bekamen.
\end{enumerate}