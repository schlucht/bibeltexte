\section{Übersichtsfragen zu 1. und 2. Samuel}
\begin{enumerate}
    \item textbf{Wie hiess der letzte Richter?}\\
    Der letzte Richter wae Samuel. Die Israeliten wollten jetzt eine König wie alle anderen Völker rund um sie auch einen haben.
    \item textbf{Wie hiessen die beiden Könige unter Samuel?}\\
    Samuel hat dann unter protest zwei Könige gesalbt. Zuerst Saul, der nicht Gottes sondern seine eigenen Wege gehen wollt. Dann wurde noch während der Regierungszeit Sauls David zum König gesalbt.
    \item textbf{Welchen Einfluss hatte die Bundeslade auf Israel und die Philister?}
    Auf Israel: Keinen. Die Bundeslade wurde nur noch als Mittel zum zweck benutzt. War Krieg sollte die Lade sie beschützen.

    Auf die Philister, hatte die Lade eine verherende Auswirkung. Ihnen zeigte sich Gott als einzig wahren Gott. Ihre Götter vielen vor der Lade um. Es brach in dem Ort der Bundeslade Krankeit und Seuchen aus.
    \item textbf{Welches waren die politischen und geistlichen Höhepunkte unter David?}
    Ruhe von allen Feinden rings um Israel. Jerusalem wurde zur Hauptstadt Israels. Ausserdem hat er den Tempelbau vorbereitet. Selber sollte er den Tempel nicht mehr bauen sondern Sohn Salomon sollte dies tun:
    \begin{bibeltext}{Sch2}{1Sam}{7.12}
        Wenn deine Tage erfüllt sind und du bei deinen Vätern liegst, so will ich deinen Samen nach dir erwecken, der aus deinem Leib kommen wird, und ich werde sein Königtum befestigen. Der wird meinem Namen ein Haus bauen, und ich werde den Thron seines Königreichs auf ewig befestigen.
    \end{bibeltext}
    \item textbf{Wie hiessen die beiden herausragenden     Sünden im Leben Davids und die daraus folgenden Gerichte Gottes?}
    Der Ehebruch mit Bathseba und die Ermordung von Uria dem Ehemann der Bathseba
\end{enumerate}