\section{Übersichtsfragen zu 5. Mose}
\begin{enumerate}
    \item Warum ist die hebräische Bezeichnung Devarim, übersetzt «Worte», für 5. Mose treffend?\\
    Im Griechischen wird 5. Mose \flqq Deuteronomium\frqq{} genannt, übersetzt \flqq Zweites Gesetz\frqq . Eigentlich ein unglücklicher Name, denn es ist kein zweites Gesetz, sondern der Rückblick eines alt gewordenen Mannes. Mose ist zu dieser Zeit mindestens doppelt so alt wie die ältesten Israeliten (mit ausnahme von Josua und Kaleb). Er lässt sein Leben Revue passieren. Ein Leben voll Abenteuer, Segen, Bewahrung, Leid und menschliches Versagens. Doch über allem stand immer die Treue und Grösse Gottes. In 5. Mose erinnert Mose seine Enkel noch einmal an alle Wundertaten Gottes, an Seine Bestimmungen und an Sein Gesetz. Dabei motiviert er das Volk bei seinem Gott zu bleiben und Gottes Anordnungen Folge zu leisten. Gleichzeitig richtet er ihr Augenmerk auch hin in die ferne Zukunft, indem er sagt: ""
    \begin{bibeltext}{Sch2}{Deut}{18:15}
        Einen Propheten wie mich wird dir der \herr, dein Gott, erwecken aus deiner Mitte, aus deinen Brüdern; auf ihn sollst du hören!
    \end{bibeltext}
    Damit zeigt Mose, dass die 5 Bücher Mose nicht das Ende, sondern der Anfang der Erlösung ist, dessen Zielkoordinaten sich in Jesus Christus treffen.
\end{enumerate}