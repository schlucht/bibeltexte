
\begin{enumerate}
    \item \textbf{Warum wird 2. Mose auch das «Buch der Erlösung» genannt?}\\
    Es zeigt die Befreiung Israels aus der Knechtschaft Ägyptens durch das Blut eines Lammes. Durch die zehn Gebote wird Gottes Willen und Heiligkeit weiter offenbart. Aber kein Mensch ist fähig diese zehn Gebote vollständig einzuhalten. Gott zeigt uns einen Ausweg durch den Weg und Dienst in der Stiftshütte. Dies ist ein wunderbarer Prohetischer hinweis aus auf den Erlöser Jesus Christus. In 2. Mose bedeutet Erlösung auch: Wiederherstellung und ganz machen. Gott macht aus einer Familie ein Volk, gibt ihnen eine Regierung, ein Gottesdienst, ein Gesetz und bestätigt ein Land. So erlöst Gott und stellt es unter seine Leitung und Fürsorge.
    \item \textbf{Worauf weist der hebräiche Titel von 3. Mode, Leviticus, hin?}\\
    Auf die Aufgabe der Priester und Leviten. Gott beruft sie in Seinen Dienst. Gott zeigt dem Volk durch den Dienst und die Aufgaben der Priester sowie durch die entsprechenden Opfer den WEg zur Gemeinschaft mit Ihm. Damit wird deutlich, dass Gottes Einladung gilt: das Volk Israel kann kommen. Jedosch nur auf dem von Ihm vorgezeichneten Weg. Dies erinnnert an die Aussage Jesu:
    \begin{bibeltext}{ELB}{Joh}{14:6}
        Ich bin der Weg, die Wahrheit und das Leben, niemand kommt zum Vater als nur durch Mich.
    \end{bibeltext}
    \item \textbf{Warum heisst 4. Mose auch "In der Wüste"?}
    Gott hat Sein Volk mit starker Hand erlöst. Aus Ihnen ein Volk geformt. Sich ihnen in wunderbarer und herrlicher Weise offenbart. Doch Israel reagiert immer wieder mit Unglauben und Auflehnung, bis das "Fass überläuft", bis zum "Point of Return". Siehe:
    \begin{bibeltext}{Sch2}{Num}{14:11}
        Und der \herr{} sprach zu Mose: \flqq Wie lange noch will mich dieses Volk verachten? Und wie lange noch wollen sie nicht an mich glauben, trotz aller Zeichen, die ich unter ihnen getan habe?\frqq{}
    \end{bibeltext}
    \begin{bibeltext}{Sch2}{Num}{14:16-22}
        Weil der \herr{} dieses Volk nicht in das Land bringen konnte, das er ihnen zugeschworen hatte, darum hat er sie in der Wüste hingeschlachtet! So lass nun die Macht des \herr n gross werden, wie du gesprochen und verheissen hast: \flqq Der \herr{} ist langsam zum Zorn und gross an der Gnade; er vergibt Schuld und Übertretungen, obgleich er keineswegs ungestraft lässt, sondern die Schuld der Väter heisucht an den Kindern, bis in das dritte und vierte Glied. Vergib nun die Schuld dieses Volkes nach deiner grossen Gnade, wie du auch diesem Volk verziehen hast non Ägypten an bis hierher.\frqq{}

        Da sprach der \herr : Ich habe vergeben nach deinem Wort. Aber - so war ich lebe und die ganze Erde mit der Herrlichkeit des \herr n erfüllt werden soll. Keiner der Männer, die meine Herrlichkeit und meine Zeichen gesehen haben, die ich in Ägypten und in der Wüste getan habe, und die mich nun schon zehnmal versucht und meiner Stimme nicht gehorcht haben.
    \end{bibeltext}
    Entsprechend beginnt eine 40-jährige Wüstenwanderung voller Not und Elend, bis alle, die Gottes Wege verwarfen, in der Wüste starben und somit das Land der Verheissung nicht zu sehen bekamen.
    \item \textbf{Warum ist die hebräische Bezeichnung Devarim, übersetzt "Wort", für 5. Mose treffend?}
    Im Griechischen wird 5. Mose "Deutoronium" genannt, übersetzt "Zweites Gesetz". Eigentlich ein unglücklicher Name, denn es ist kein zweites Gesetz, sondern der Rückblick eines alt gewordenen Mannes. Mose ist zu dieser Zeit mindestens doppelt so alt wie die ältesten Israeliten (Ausnahme Kaleb und Josua). Er lässt sein Leben Revue passieren. Ein Leben voll Abenteuer, Segen, Bewahrung, Leid und menschliches Versagens. Docj über allem stand immer die Treue unf Grösse Gottes. In 5. Mose erinnert Mose "seine Enkel" noch einmal an alle Wundertaten Gottes, an Seine Bestimmungen und an Sein Gesetz. Dabei motiviert er das Volk bei seinem Gott zu bleiben und Gottes Anordungen Folge zu leisten.

    Gleichzeitig richtet er ihr Augenmerk auch hin in die ferne Zukunft, indem er sagt:
    \begin{bibeltext}{Sch2}{Deu}{18:5}
        Einen Propheten wie mich wir der \herr , dein Gott, dir erwecken aus dir und aus deinen Brüdern; dem sollt ihr gehorchen.
    \end{bibeltext}
    Damit zeigt Mose, dass die 5 Bücher Mose nicht das Ende, sondern der Anfang der Erlösung ist, \textbf{dessen Zielkoordinaten sich in Jesus Christus treffen.}

\end{enumerate}