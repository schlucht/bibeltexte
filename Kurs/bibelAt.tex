\chapter{Das Alte Testament}
\section{Allgemeines zum Buch 1. Mose}
\subsection{WER?}
Als Autor wird Moses erwähnt. Dies wird in mehreren Schriftstellen in der Bibel bestätigt.
\utitle{Aus dem Alten Testament: }
\begin{bibeltext}{Sch2}{Ex}{17:14}
	Da sprach der \herr{} zu Mose: \textquote{Schreibe das zum Gedenken in ein Buch und präge es den Ohren Josuas ein: Ich will das Andenken Amaleks ganz und gar austilgen unter dem Himmel.}
\end{bibeltext}
\begin{bibeltext}{Sch2}{Num}{33:2}
    Und Mose schrieb ihren Auszug und ihre Tagereisen auf Befehl des \herr{} nieder.
\end{bibeltext}

\begin{bibeltext}{Sch2}{Neh}{13:1}
    Zu jener Zeit wurde den Ohren des Volkes im Buch Moses gelesen und darin geschrieben gefunden, dass kein Ammoniter und Moabiter in die Geneinde Gottes kommen sollte ewiglich.
\end{bibeltext}
\utitle{Aus dem neuen Testament: }
\begin{bibeltext}{Sch2}{Mt}{8:4}
    Und Jesus spricht zu ihm: \textquote{Sieh zu, dass du es niemand sagst; sondern geh hin, zeige dich dem Priester und bringe das Opfer dar, das Mose befohlen hat, ihnen zum Zeugnis!}
\end{bibeltext}
\begin{bibeltext}{Sch2}{Joh}{5:14}
    Danach findet ihn Jesus im Tempel und spricht zu ihm: \textquote{Siehe, du bist gesund geworden; sündige hinfort nicht mehr, damit dir nicht etwas Schlimmes widerfährt!}
\end{bibeltext}
\begin{bibeltext}{Sch2}{Rom}{10:19}
    Aber ich frage: \textquote{Hat es Israel nicht erkannt? Schon Mose sagt: \textquote{Ich will euch zur Eifersucht reizen durch das was kein Volk ist; durch ein unverständiges Volk will ich euch erzürnen}}
\end{bibeltext}
\begin{bibeltext}{Sch2}{2Kor}{3:15}
   Doch bis zum heutigen Tag liegt die Decke auf ihrem Herzen, sooft Mose gelesen wird. 
\end{bibeltext}
In diesen Bibletexten wird bestätigt, dass das die Bücher Mose von Moses selber geschrieben wurde. Auf Grund seiner hohen Stellung im Hofe des Pharao hat er eine Bildung genossen, die ihm das Schreiben ermöglichte. Bis jetzt wurde noch keine Beweise aufgeführt oder gefunden, die Mose als Verfasser widerlegen.
\subsection{WEM?}
Eigentlich wurde das Buch für die ganze Menschheit geschrieben. Das erste Buch Mose (Genesis) bezeugt, dass Gott existiert. Dies wir schon im ersten Satz des ersten Kapitel erwähnt:
\begin{bibeltext}{Sch2}{Gen}{1:1}
    Im Anfang schuf Gott die Himmel und die Erde.
\end{bibeltext}
Es gibt keine Anzeichen, dass Gott irgendwie entstanden ist. Der \herr{} war schon immer. Er ist der Anfang und das Ende. Das $\alpha$ und das $\Omega$.
Die Bücher Mose wurden in erster Linie für Israel geschrieben. Dies sag der \herr{} auch Josua im ersten Kapitel.
\begin{bibeltext}{Sch2}{Jos}{1:8}
   Lass dieses Buch des Gesetzes nicht von deinem Mund weichen, sondern erforsche darin Tag und Nacht, damit du darauf achtest, alles zu befolgen, was darin geschrieben steht; denn dann wirst du Gelingen haben auf deinen Wegen, und dann wirst du weise handeln. 
\end{bibeltext}
Das erste Buch Mose enthält die Geschichte der Welt und vom Volk Israel und die restlichen viel Teile beinhalten die Gesetzte und Anweisung von Gott an das Volk Israel.
\subsection{WANN?}
Die Bücher Mose sind wohl gegen Ende der Wüstenwanderung geschrieben worden. Der Auszug aus Ägypten wird ca. 1445 v.Chr. datiert. Moses Tod war um die Zeit 1405 v.Chr.. Irgend wann in dieser Zeit wurden die 5 Bücher geschrieben. Das erste Buch Mose wurde wohl zu beginn des Exodus geschrieben.
\subsection{WAS?}
Das erste Buch Mose umfasst 2000 Jahre Urgeschichte. Sie beginnt bei der Schöpfung und endet mit Tod Josef in Ägypten. 400 Jahre danach erscheint dann Mose.
Das erste Buch ist in zwei Teilen unterteilt und diese jeweils wieder in vier Unterteile. Schon im ersten Buch der Bibel kann man den Heilsplan Gottes erkennen. Nach dem Sündenfall des Menschen, beginnt ein systematischer Plan Gottes um die Menschen wieder auf den richtigen Weg in seine nähe zu kriegen. Dieser Heilsplan die Erlösung, d.h. die Rückführung des Menschen zu Gott durchzieht die Schrift wie ein roter Faden
\begin{enumerate}
    \item \textbf{die Urgeschichte \hspace{1cm} \bibleverse{Gen} {1-11}}
    
    \begin{tabular}{cll}
        1.&die Schöpfung &\bibleverse{Gen} {1-2} \\
        2.&der Sündenfall &\bibleverse{Gen} {3-5} \\
        3.&die Sintflut &\bibleverse{Gen} {6-9} \\
        4.&die Sprachenverwirrung&\bibleverse{Gen} {10.11}\\ 
    \end{tabular}
    \item \textbf{die Patriarchengeschichte \hspace{1cm} \bibleverse{Gen} {1-11}}
    
    \begin{tabular}{cll}
        1.&Abraham& \bibleverse{Gen} {12;1-25;8}\\
        2.&Isaak &\bibleverse{Gen} {21;1-35:29}\\
        3.&Jakob &\bibleverse{Gen} {25;21-50;14}\\
        4.&Josef &\bibleverse{Gen} {30;22-50;26}\\
    \end{tabular}
\end{enumerate}
\subsection{WIE?}
Der erste Teil ist erzählend aufgeschrieben. Es ist die Geschichte der Menschen aufgeschrieben. Der Teil vor der Sintflut und dann nach der Sintflut. Nach der Sintflut hat sicht die Welt verändert. Die Erde wurde anders die Menschen wurden jünger. Was sich aber nicht geändert hat ist der Mensch selber. Auch nach der Sintflut blieb er rebellisch und Gott fern.

Theologisch kann viel von den Patriarchen gelernt werden. Auch die Geschichte von Josef in Ägypten ist bemerkenswert und ist immer noch aktuell. Diese zeigt auf das Gott für jeden Menschen seinen Plan hat. Auch für uns. Da wir mitten drin sind sehen wir selber das nur schwer. Ich glaube so richtig klar wird es uns dann erst nach dem Tod.


\begin{enumerate}
    \item \textbf{In welche Zeitepochen bzw. Bündnisse
    lässt sich das 1. Mosebuch unterteilen?}\\
    Das erste Buch Mose wird in folgendes Bündnisse eingeteilt:
    \begin{itemize}
        \item Schöpfung
        \item Sündenfall
        \item Noah
        \item Turmbau von Babel
        \item Abraham
    \end{itemize}
    \item \textbf{Welche göttliche Tatsachen zeigt uns der Schöpfungsbericht?}\\
    Gott ist präexistens und alles wurde durch ihn geschaffen. Das heisst er hat vor der Erschaffung der Welt schon existiert.
    \item \textbf{Was ist die Krönung der Schöpfung Gottes?}\\
    Die Krönung der Schöpfung ist die Erschaffung des Menschen am 6 Tag nach seinem Ebenbild (1. Mose 1,27). Er schuf sie als Mann und Frau.
    \item \textbf{Wo offenbart sich Gottes Heilsplan zum ersten Mal?}
    \begin{bibeltext}{Sch2}{Gen}{3:15}
        Und ich will feindschaft setzen zwischen dir und der Frau, zwischen deinem Samen und ihrem Samen: Er wird dir den Kopf zertreten und du wirst ihn in die Ferse stechen.
    \end{bibeltext}
    \item \textbf{Wie offenbart sich Gottes Heilsplan durch Abraham?}
    \begin{bibeltext}{Sch2}{Gen}{12:3}
        ...in dir sollen gesegnet werden alle Geschlechter auf der Erde.
    \end{bibeltext}
    \item \textbf{Welche drei Haupt-Verheissungen gibt Gott Abraham }
        \begin{itemize}
            \item 1Mo 12,1-3
            \item 1Mo 15,12-18
        \end{itemize}
        Einen ewigen betreffend Volk und Land sowie segen für alle Nationen
    \item \textbf{Wie lautet die genealogische Linie hin zum verheissenen Erlöser?}
    \begin{itemize}
        \item Abraham
         \begin{bibeltext}{Sch2}{Gen}{12:3}
            Ich will segnen, die  dich segnen, und verfluchen, die dir fluchen: und in dir sollen gesegnet werden alle Geschlechter auf der Erde.
        \end{bibeltext}
        \begin{bibeltext}{Sch2}{Gen}{17:7}
            Und ich will einen Bund aufrichten zwischen mir und dir und deinem Samen nach dir von Geschlecht zu Geschlecht als ewigen Bund, dein Gott zu sein und der deines Samens nach dir. 
        \end{bibeltext}
        \begin{bibeltext}{Sch2}{Gen}{17:16}
            denn ich will sie [Sarah] segnen und will dir auch von ihr einen Sohn geben. Ich will sie segnen, und sie soll Nationen werden, und Könige von Völkern sollen von ihr kommen!
        \end{bibeltext}
        \begin{bibeltext}{Sch2}{Gen}{22:18}
            und in deinem Samen sollen alle Völker der Erde gesegnet werden, weil du meiner Stimme gehorsam warst!
        \end{bibeltext}
        \item Isaak
        \begin{bibeltext}{Sch2}{Gen}{17:19}
            Da sprach Gott: Nein, sondern Sarah, deine Frau, soll dir einen Sohn gebären, den sollst du Isaak nennen: denn ich will ihm einen Bund aufrichten als einen ewigen Bund für seinen Samen nach ihm.
        \end{bibeltext}
        \begin{bibeltext}{Sch2}{Gen}{25:11}
            Und es geschah nach dem Tod Abrahams, da segnete Gott seinen Sohn Isaak. Und Isaak wohnte bei den \glqq Brunnen des Lebendigen, der sieht\grqq.
        \end{bibeltext}
        \item Jakob
        \begin{bibeltext}{Sch2}{Gen}{27:28-29}
            Gott gebe dir vom Tau des Himmels und vom fettesten Boden und Korn und Most in Fülle! Völker sollen dir dienen und Geschlechter sich vor dir beugen; sein ein Herr über deine Brüder, und die Söhne deiner Mutter sollen sich vor dir beugen. Verflucht sei, wer dir flucht, und gesegnet sei, wer dich segnet!
        \end{bibeltext}
        \begin{bibeltext}{Sch2}{Gen}{28:13-15}
            Und siehe, der \herr stand über ihr und sprach: Ich bin der \herr, der Gott deines Vaters Abraham und der Gott Isaak; das Land auf dem du liegst, will ich dir und deinem Samen geben. Und dein Same soll werden der Staub der Erde, und nach Westen, Osten, Norden und Süden sollst du dich ausbreiten; und dir und in deinem Samen sollen gesegnet werden alle Geschlechter der Erde! Und siehe, ich bin mit dir, und ich will dich behüten überall, wo du hinziehst, und dich wieder in dieses Land bringen. Denn ich will dich nicht verlassen, bis ich vollbracht habe, was ich dir zugesagt habe!
        \end{bibeltext}
        \begin{bibeltext}{Sch2}{Gen}{35:9-10}
            Und Gott erschien Jakob zum zweiten Mal, seit er aus Paddan-Aram gekommen war, und segnete ihn. Und Gott sprach: Dein Name ist Jakob, aber du sollst nicht mehr Jakob heissen, sondern Israel soll dein Name sein! Und so gab er ihm den Namen Israel!
        \end{bibeltext}
        \item Juda
        \begin{bibeltext}{Sch2}{Gen}{49:8-12}
            Dich Juda, werden deine Brüder preisen! Deine Hand wird auf dem Nacken deiner Feinde sein; vor dir werden sich die Söhne deines Vaters beugen. Juda ist ein junger Löwe; mit Beute beladen steigst du, mein Sohn, empor! Er hat sich gekauert und gelagert wie ein Löwe, wie eine Löwin; wer darf ihn aufwecken? Es wird das Zepter nicht von Juda weichen, noch der Herrscherstab von seinen Füssen, bis der Schilo kommt, und ihm werden die Völker gehorsam sein. Er wird sein Füllen an den Weinstock binden und das Junge seiner Eselin an die Edelrebe; er wird sein Kleid im Wein waschen und seinen Mantel in Traubenblut; seine Augen sind dunkler als Wein und seine Zähne weisser als Milch.
        \end{bibeltext}
    \end{itemize}
\end{enumerate}
%\subsubsection{2. Mose 1 (Exodus 1)}
%\input{Kurs/bibelkommentar/1_Moses/2.tex}
%\input{Kurs/bibelkommentar/1_Moses/3.tex}
%\input{Kurs/bibelkommentar/1_Moses/4.tex}
%\input{Kurs/bibelkommentar/1_Moses/5.tex}
\input{Kurs/bibelkommentar/1_Moses/6.tex}
%\subsubsection{1. Mose 7 (Genesis 7)}
In diesem Kapitel geht es in die Arche. Jetzt fordert Gott Noah auf das Schiff zu füllen und selber in die Arche zu steigen. Gott schickt ein Pärchen von jeder Tierart. 
%\input{Kurs/bibelkommentar/1_Moses/8.tex}
%\subsection{Kapitel 9}

Jetzt nachdem sie die Arche verlassen haben und wieder an Land sind, segnet Gott Noah und seine Söhne. Als erstes sollen sie sich wieder vermehren und die Erde bevölkern. Das Töten war in der Urfassung der Schöpfung nicht vorgesehen. Hier stellt Gott alles was auf der Erde lebt den Menschen Untertan. 

%\input{Kurs/bibelkommentar/1_Moses/10.tex}


\begin{enumerate}
    \item \textbf{Warum wird 2. Mose auch das «Buch der Erlösung» genannt?}\\
    Es zeigt die Befreiung Israels aus der Knechtschaft Ägyptens durch das Blut eines Lammes. Durch die zehn Gebote wird Gottes Willen und Heiligkeit weiter offenbart. Aber kein Mensch ist fähig diese zehn Gebote vollständig einzuhalten. Gott zeigt uns einen Ausweg durch den Weg und Dienst in der Stiftshütte. Dies ist ein wunderbarer Prohetischer hinweis aus auf den Erlöser Jesus Christus. In 2. Mose bedeutet Erlösung auch: Wiederherstellung und ganz machen. Gott macht aus einer Familie ein Volk, gibt ihnen eine Regierung, ein Gottesdienst, ein Gesetz und bestätigt ein Land. So erlöst Gott und stellt es unter seine Leitung und Fürsorge.
    \item \textbf{Worauf weist der hebräiche Titel von 3. Mode, Leviticus, hin?}\\
    Auf die Aufgabe der Priester und Leviten. Gott beruft sie in Seinen Dienst. Gott zeigt dem Volk durch den Dienst und die Aufgaben der Priester sowie durch die entsprechenden Opfer den WEg zur Gemeinschaft mit Ihm. Damit wird deutlich, dass Gottes Einladung gilt: das Volk Israel kann kommen. Jedosch nur auf dem von Ihm vorgezeichneten Weg. Dies erinnnert an die Aussage Jesu:
    \begin{bibeltext}{ELB}{Joh}{14:6}
        Ich bin der Weg, die Wahrheit und das Leben, niemand kommt zum Vater als nur durch Mich.
    \end{bibeltext}
    \item \textbf{Warum heisst 4. Mose auch "In der Wüste"?}
    Gott hat Sein Volk mit starker Hand erlöst. Aus Ihnen ein Volk geformt. Sich ihnen in wunderbarer und herrlicher Weise offenbart. Doch Israel reagiert immer wieder mit Unglauben und Auflehnung, bis das "Fass überläuft", bis zum "Point of Return". Siehe:
    \begin{bibeltext}{Sch2}{Num}{14:11}
        Und der \herr{} sprach zu Mose: \flqq Wie lange noch will mich dieses Volk verachten? Und wie lange noch wollen sie nicht an mich glauben, trotz aller Zeichen, die ich unter ihnen getan habe?\frqq{}
    \end{bibeltext}
    \begin{bibeltext}{Sch2}{Num}{14:16-22}
        Weil der \herr{} dieses Volk nicht in das Land bringen konnte, das er ihnen zugeschworen hatte, darum hat er sie in der Wüste hingeschlachtet! So lass nun die Macht des \herr n gross werden, wie du gesprochen und verheissen hast: \flqq Der \herr{} ist langsam zum Zorn und gross an der Gnade; er vergibt Schuld und Übertretungen, obgleich er keineswegs ungestraft lässt, sondern die Schuld der Väter heisucht an den Kindern, bis in das dritte und vierte Glied. Vergib nun die Schuld dieses Volkes nach deiner grossen Gnade, wie du auch diesem Volk verziehen hast non Ägypten an bis hierher.\frqq{}

        Da sprach der \herr : Ich habe vergeben nach deinem Wort. Aber - so war ich lebe und die ganze Erde mit der Herrlichkeit des \herr n erfüllt werden soll. Keiner der Männer, die meine Herrlichkeit und meine Zeichen gesehen haben, die ich in Ägypten und in der Wüste getan habe, und die mich nun schon zehnmal versucht und meiner Stimme nicht gehorcht haben.
    \end{bibeltext}
    Entsprechend beginnt eine 40-jährige Wüstenwanderung voller Not und Elend, bis alle, die Gottes Wege verwarfen, in der Wüste starben und somit das Land der Verheissung nicht zu sehen bekamen.
    \item \textbf{Warum ist die hebräische Bezeichnung Devarim, übersetzt "Wort", für 5. Mose treffend?}
    Im Griechischen wird 5. Mose "Deutoronium" genannt, übersetzt "Zweites Gesetz". Eigentlich ein unglücklicher Name, denn es ist kein zweites Gesetz, sondern der Rückblick eines alt gewordenen Mannes. Mose ist zu dieser Zeit mindestens doppelt so alt wie die ältesten Israeliten (Ausnahme Kaleb und Josua). Er lässt sein Leben Revue passieren. Ein Leben voll Abenteuer, Segen, Bewahrung, Leid und menschliches Versagens. Docj über allem stand immer die Treue unf Grösse Gottes. In 5. Mose erinnert Mose "seine Enkel" noch einmal an alle Wundertaten Gottes, an Seine Bestimmungen und an Sein Gesetz. Dabei motiviert er das Volk bei seinem Gott zu bleiben und Gottes Anordungen Folge zu leisten.

    Gleichzeitig richtet er ihr Augenmerk auch hin in die ferne Zukunft, indem er sagt:
    \begin{bibeltext}{Sch2}{Deu}{18:5}
        Einen Propheten wie mich wir der \herr , dein Gott, dir erwecken aus dir und aus deinen Brüdern; dem sollt ihr gehorchen.
    \end{bibeltext}
    Damit zeigt Mose, dass die 5 Bücher Mose nicht das Ende, sondern der Anfang der Erlösung ist, \textbf{dessen Zielkoordinaten sich in Jesus Christus treffen.}

\end{enumerate}
\section{Übersichtsfragen zu 3. Mose}
\begin{enumerate}
    \item Worauf weist der hebräische Titel von 3. Mose, Leviticus, hin?\\
    Auf die Aufgabe der Priester und Leviten. Gott beruft sie in Seinen Dienst. Gott zeigt dem Volk durch den Dienst und die Aufgaben der Priester sowie durch die entsprechenden Opfer den Weg zur Gemeinschaft mit Ihm. Damit wird deutlich, dass Gottes Einladung gilt: das Volk Israel kann kommen. Jedoch nur auf dem von Ihm vorgezeichneten Weg. Dies erinnert an die Aussage Jesu:
\begin{bibeltext}{Sch2}{Joh}{14:6}
    Jesus spricht zu ihm: Ich bin der Weg und die Wahrheit und das Leben; niemand kommt zum Vater als nur durch mich!
\end{bibeltext}
\end{enumerate}
\section{Übersichtsfragen zu 4. Mose}
\begin{enumerate}
    \item textbf{Warum heisst 4. Mose auch «In der Wüste»?}\\
    Gott hat sein Volk mit starker Hand erlöst. Aus ihnen ein Volk geformt. Sich ihnen und wunderbarer und herrlicher Weise offenbart. Doch Isreal reagiert immer wieder in Unglauben und in Auflehnung bis das \flqq bis das Fass überläuft \frqq, bis zum \flqq Point of no return \frqq.
    \begin{bibeltext}{Sch2}{Num}{14:11}
        Und der \herr sprach zu Mose: Wie lange noch will mich dieses Volk verachten? Und wie lange noch wollen, sie nicht an mich glauben, trotz aller Zeichen, die ich unter ihnen getan habe?
    \end{bibeltext}
    \begin{bibeltext}{Sch2}{Num}{14:22-23}
        Keiner der Männer die meine Herrlichkeit und meine Zeichen gesehen haben, die ich in Ägypten und in der Wüste getan habe, und die mich nun schon zehnmal versucht und meiner Stimme nicht gehorcht haben, soll das Land sehen, das ich ihren Vätern zugeschworen habe; ja keiner soll es sehen, der micht verachtet hat!
    \end{bibeltext}
    Entsprechent beginnt eine 40jährige Wüstenwanderung, voll Not und Elend, bis alle, Gottes Wege verwarfen, in der Wüsten starben und somit das Land der Verheissung nicht zu sehen bekamen.
\end{enumerate}
\section{Übersichtsfragen zu 5. Mose}
\begin{enumerate}
    \item Warum ist die hebräische Bezeichnung Devarim, übersetzt «Worte», für 5. Mose treffend?\\
    Im Griechischen wird 5. Mose \flqq Deuteronomium\frqq{} genannt, übersetzt \flqq Zweites Gesetz\frqq . Eigentlich ein unglücklicher Name, denn es ist kein zweites Gesetz, sondern der Rückblick eines alt gewordenen Mannes. Mose ist zu dieser Zeit mindestens doppelt so alt wie die ältesten Israeliten (mit ausnahme von Josua und Kaleb). Er lässt sein Leben Revue passieren. Ein Leben voll Abenteuer, Segen, Bewahrung, Leid und menschliches Versagens. Doch über allem stand immer die Treue und Grösse Gottes. In 5. Mose erinnert Mose seine Enkel noch einmal an alle Wundertaten Gottes, an Seine Bestimmungen und an Sein Gesetz. Dabei motiviert er das Volk bei seinem Gott zu bleiben und Gottes Anordnungen Folge zu leisten. Gleichzeitig richtet er ihr Augenmerk auch hin in die ferne Zukunft, indem er sagt: ""
    \begin{bibeltext}{Sch2}{Deut}{18:15}
        Einen Propheten wie mich wird dir der \herr, dein Gott, erwecken aus deiner Mitte, aus deinen Brüdern; auf ihn sollst du hören!
    \end{bibeltext}
    Damit zeigt Mose, dass die 5 Bücher Mose nicht das Ende, sondern der Anfang der Erlösung ist, dessen Zielkoordinaten sich in Jesus Christus treffen.
\end{enumerate}
\section{Übersichtsfragen zu 1. Chronik}
\begin{enumerate}
    \item Fragen
\end{enumerate}
\section{Übersichtsfragen zu 1. und 2. Samuel}
\begin{enumerate}
    \item \textbf{Wie hiess der letzte Richter?}\\
    Der letzte Richter wae Samuel. Die Israeliten wollten jetzt eine König wie alle anderen Völker rund um sie auch einen haben.
    \item \textbf{Wie hiessen die beiden Könige unter Samuel?}\\
    Samuel hat dann unter protest zwei Könige gesalbt. Zuerst Saul, der nicht Gottes sondern seine eigenen Wege gehen wollt. Dann wurde noch während der Regierungszeit Sauls David zum König gesalbt.
    \item \textbf{Welchen Einfluss hatte die Bundeslade auf Israel und die Philister?}
    Auf Israel: Keinen. Die Bundeslade wurde nur noch als Mittel zum zweck benutzt. War Krieg sollte die Lade sie beschützen.

    Auf die Philister, hatte die Lade eine verherende Auswirkung. Ihnen zeigte sich Gott als einzig wahren Gott. Ihre Götter vielen vor der Lade um. Es brach in dem Ort der Bundeslade Krankeit und Seuchen aus.
    \item \textbf{Welches waren die politischen und geistlichen Höhepunkte unter David?}
    Ruhe von allen Feinden rings um Israel. Jerusalem wurde zur Hauptstadt Israels. Ausserdem hat er den Tempelbau vorbereitet. Selber sollte er den Tempel nicht mehr bauen sondern Sohn Salomon sollte dies tun:
    \begin{bibeltext}{Sch2}{1Sam}{7:12}
        Wenn deine Tage erfüllt sind und du bei deinen Vätern liegst, so will ich deinen Samen nach dir erwecken, der aus deinem Leib kommen wird, und ich werde sein Königtum befestigen. Der wird meinem Namen ein Haus bauen, und ich werde den Thron seines Königreichs auf ewig befestigen.
    \end{bibeltext}
    \item \textbf{Wie hiessen die beiden herausragenden     Sünden im Leben Davids und die daraus folgenden Gerichte Gottes?}
    Der Ehebruch mit Bathseba und die Ermordung von Uria dem Ehemann der Bathseba
\end{enumerate}
\subsection{Übersichtsfragen zu Hiob}
\begin{enumerate}
    \item \textbf{Wie lauten einige Charaktereigenschaften Hiobs}\\
    Untadelig, rechtschaffen, gottesfürchtig, geistlich gesinnt, keusch, hilfsbereit
    \item \textbf{Warum muss Hiob leiden?}\\
    Es tobt ein Kampf zwischen Gott und den Teufel. Der ist der Meinung, dass sobald es Hiob schlecht gehen wird dieser vom Glauben abfallen wird.
    \item \textbf{Wie lassen sich die ersten sechs Prüfungen Hiobs zusammenfassen?}\\
    Vernichtung seines gesamten Besitzes, Zerstörung seiner Gesundheit, Ratschlag seiner Ehefrau sich von Gott loszusagen und Selbstmord zu begehen.
     \item \textbf{Was führte zum Zerbruch von Hiob?}\\
     Die siebte Prüfung: Seine takt- und lieblosen, besserwisserischen und verurteilenden Freunde.
     \item \textbf{Was lernen wir aus dem Buch Hiob?}\\
     \begin{itemize}
     		\item Gott beanwortet nicht alle \flqq Warum-Fragen \frqq
     		\item Gott ist würdig, unabhängig von allen Lebensumstaänden angebetet zu werden
     \end{itemize}
     \item \textbf{Was bewirkt die grosse Wende?}\\
     Die Begegnung mit Gott, gefolgt von Busse und Beugung; das Gebet für seine Freunde war der Auslöser für vollständige Wiederherstellung.
     \item \textbf{Wo und wann fand die Geschichte Hiobs statt?}\\
     Im Land Uz, in Edom, was dem heutigen Jordanien entspricht, im 3. Jahrtausend vor Christus.
     
\end{enumerate}



